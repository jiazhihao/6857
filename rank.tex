\section{Ranking}
\label{s:rank}

Despite that \sys may report a long list of bugs, we are particularly
interested in integer errors that developers are willing to fix.
We consider an integer error as a serious bug if the result is
either controlled from untrusted sources or used at sensitive sinks.
Such sources and sinks are generally specific to the system being
analyzed.  This section discuss them in the context of the Linux
kernel.

Untrusted sources are essentially all channels into the kernel from
the outside, including data copied from user space (via system
calls, virtual file systems like \cc{sysfs}, or kernel functions
like \cc{copy_from_user}), received from network, and read from
disk.
%\xw{Xen: Dom0 input from DomU?}
The kernel usually needs to perform sanity checks on these data to
avoid being tricked by an adversary.

Two most notable error-prone channels are \cc{ioctl} and \cc{setsockopt},
due to their infamous irregular interfaces and pervading implementations
in device drivers and network protocols.  \sys pays special attention
to integer errors that could happen in their calling context.

In general, hardware devices are considered trusted.  Driver code
often does not validate the response data from hardware.  According
to our initial experience, the driver developers are unlikely to
accept patches that fix the validation.  Thus, \sys filters out
integer errors caused by hardware,
%\sys also trusts kernel module parameters for the same reason.
%
with the exception of USB devices.  Recent research has demonstrated
how to develop a malicious USB device to plug in a Linux box and
exploit the kernel~\cite{usb:buffer-overflow}.

Sensitive sinks include branching conditions that make logic
decisions, data escaping from the kernel to the outside, and data
sizes for allocation and I/O communication.  An erroneous integer
value used in such places are more likely to damage the system.

\sys requires annotations to specify these sources and sinks.
Currently we have 55 annotations for the Linux kernel.
