\section{Ranking}

To make \sys useful, we are particularly interested in integer
errors that developers are willing to fix, despite that \sys may
report a long list of bugs.

\sys considers an integer error as a serious bug if the result is
either controlled from untrusted sources or used at sensitive sinks.
Such sources and sinks are generally specific to the system being
analyzed.  This section discuss them in the context of the Linux
kernel.

Untrusted sources are essentially all channels into the kernel from
the outside, including data copied from user space (via system
calls, virtual file systems like \cc{sysfs}, or kernel functions
like \cc{copy_from_user}), received from network, and read from
disk. [[[Xen: Dom0 input from DomU?]]]
The kernel usually needs to perform sanity checks on these data to
avoid being tricked by an adversary.

Two most notable error-prone channels are \cc{ioctl} and \cc{setsockopt},
due to their infamous irregular interfaces and pervading implementations
in device drivers and network protocols.  \sys pays special attention
to integer errors that could happen in their calling context.

\sys currently trusts hardware devices, other it would report too
many errors while the kernel developers are unlikely to accept such
patches based on our initial experience.  \sys also trusts kernel
module parameters for the same reason.
[[[make an exception for USB devices?]]]

Sensitive sinks include branching conditions that make logic
decisions, data escaping from the kernel to the outside, and data
sizes for allocation and I/O communication.  An erroneous integer
value used in such places are more likely to damage the system.

\sys requires annotations to specify untrusted sources and sensitive
sinks.  Currently we have annotated XXX for the Linux kernel.
