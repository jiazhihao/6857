\section{Design}
\label{s:gen}

This section describes \sys's design for constraint generation and
solving.

\subsection{Transformations}

\sys performs a series of code transformations for better constraint
generation.  As discussed in \autoref{s:goal}, these transformations
should not destroy the semantics of integer security.

\paragraph{Value equality testing.}
How to determine the values from two load instructions
are the same? Load hoisting, unsound aliasing rules.

\autoref{f:hoist} shows in the hoisting algorithm.

\begin{figure}
\footnotesize
\begin{algorithmic}
\footnotesize
\Procedure{Hoist}{$I$}\Comment{$I$ is a load instruction}
\State $\mathit{loc} \gets I$'s memory location to load from
\Loop
\If{\textbf{not} \Call{HoistInBlock}{$I$, $\mathit{loc}$}}
	\State \Return
\EndIf
\State $\mathit{blk} \gets$ \Call{ChooseTargetBlock}{$I$, $\mathit{loc}$}
\If{$\mathit{blk} = \textbf{nil}$}
	\State \Return
\EndIf
\State Move $I$ to the end of $blk$
\EndLoop
\EndProcedure
\\
\Function{HoistInBlock}{$I$, $\mathit{loc}$}
\Loop
\State $\mathit{prev} \gets I$'s previous instruction in current block
\If{$\mathit{prev} = \textbf{nil}$}
	\Comment{Moved to beginning of the block?}
	\State \Return \textbf{true}
\EndIf
\If{$\mathit{prev}$ may modify $\mathit{loc}$ \textbf{or} \\
\hspace{3.6em} \textbf{not} $\mathit{loc}$ dominates $\mathit{prev}$}
	\State \Return \textbf{false}
\EndIf
\State Move $I$ before $\mathit{prev}$
\EndLoop
\EndFunction
\\
\Function{ChooseTargetBlock}{$I$, $\mathit{loc}$}
\State $\mathit{blk} \gets I$'s block
\State $\mathit{anc} \gets$ the common ancestor of $\mathit{blk}$'s predecessor(s)
\If{$\mathit{anc} = \mathit{blk}$ \textbf{or} \textbf{not} $\mathit{loc}$ dominates $\mathit{anc}$}
	\State \Return \textbf{nil}
\EndIf
\State $\mathit{blkset} \gets \{\mathit{anc}\}$
\If{\Call{CanBlocksModify}{$\mathit{loc}$, $blk$, $\mathit{blkset}$}}
	\State \Return \textbf{nil}
\EndIf
\State \Return $\mathit{anc}$
\EndFunction
\\
\Function{CanBlocksModify}{$\mathit{loc}$, $\mathit{blk}$, $\mathit{blkset}$}
\ForAll{$b \in \mathit{blk}$'s predecessor(s)}
	\If{$b \notin \mathit{blkset}$}
		\State $\mathit{blkset} \gets \mathit{blkset} \cup \{b\}$
		\ForAll{$\mathit{instr} \in b$}
			\Comment{Can $b$ modify $\mathit{loc}$?}
			\If{$\mathit{instr}$ may modify $\mathit{loc}$}
				\State \Return \textbf{true}
			\EndIf
		\EndFor
		\If{\Call{CanBlocksModify}{$\mathit{loc}$, $b$, $\mathit{blkset}$}}
			\State \Return \textbf{true}
		\EndIf
	\EndIf
\EndFor
\State \Return \textbf{false}
\EndFunction

\end{algorithmic}

\caption{The hoisting algorithm to move a load instruction to the
earliest possible point within a function.  It repeats the two
phases: first try to move the instruction to the beginning of its
basic block; if successful, try to move it into the common ancestor
of the block's predecessors.}
\label{f:hoist}
\end{figure}

\paragraph{Overflow-before-check.}
\sys invokes LLVM to push every integer operation to the latest
possible point, so that the result is computed only when it is
needed.

\paragraph{Checking idiom recognition.}

\paragraph{Pointer arithmetic simplification.}
- symbolic~\cite{engelen:symbolic}.

\subsection{Error Constraint Generation}

\paragraph{Conversion constraint.}
\sys by default does not generate constraints for checking integer
conversions.  One may simply invoke GCC with \cc{-Wconversion} to
inspect potentially dangerous conversions.
[[[\sys does something for critical integers, array indices, annotated sizes.]]]

\paragraph{In-loop constraint.}
Move in-loop constraints out.

\subsection{Path Constraint Generation}

Path constraint generation.

unroll loops once.

\subsection{Value Range Inference}

Range constraint \& annotations?

\subsection{Limitations}

miss bugs in some configurations, architectures,
and assembly code.
