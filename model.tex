\section{Integer Model}
\label{s:model}

This section presents an integer model and applies it to
\autoref{f:bridge} to demonstrate how it is used to catch integer
errors.

\sys assumes two's complement~\cite[\chapterautorefname~4.2.1]{intel:vol1},
a de facto standard integer representation on modern architectures.
An $n$-bit unsigned integer is in the range $0$ to $2^n-1$, while
an $n$-bit signed integer is in the range $-2^{n-1}$ to $2^{n-1}-1$,
with the most significant bit indicating the sign.  If not particularly
indicated, $x$ and $y$ below are $n$-bit integers.  A subscript $s$
or $u$ may be pinned to an operation to indicate it operates on
signed or unsigned integers, respectively.

\subsection{Integer Error Precondition}
\label{s:precondition}

\sys adopts the CERT C secure coding guidelines for integer
operations~\cite[\chapterautorefname~5]{seacord:secure-c} as a basis
and considers violations of the guidelines as integer errors.  These
violations cover a set of integer-based vulnerabilities, sometimes
referred as arithmetic overflow (underflow, wraparound), division-by-zero,
oversized shift, and signedness bugs in the literature.  We summarize
them below.

\paragraph{Addition \& subtraction \& multiplication.}
Such an operation may lead to an integer error if the result of the
corresponding infinitely ranged mathematical operation falls out
of the predefined range.  For example, $\cc{maxnum}\times_u 16$
causes an integer error if $\cc{maxnum} = \cc{0xf0000000}$, because
the mathematical product $\cc{0xf0000000} \times 16 = 2^{32}\times
3\times 5$ is out of the range of $32$-bit unsigned integers ($0$
to $2^{32} - 1$).

\paragraph{Division.}
A division operation causes an integer error if the divisor is 0.
Additionally, signed division $x\ /_s\ y$ may lead to an integer
error if $x = -2^{n-1}$ and $y = -1$, because the mathematical
result $2^{n-1}$ is not in the range of $n$-bit signed integers
($-2^{n-1}$ to $2^{n-1}-1$).

\paragraph{Shift.}
The bitwise shifts $x \shl y$ and $x \shr y$ are considered as
integer errors if $y \geq_u n$, which is undefined according to the
C standard.  For example, $(1 \shl 32)$ may yield 0 (e.g., on x86)
or 1 (e.g., on PowerPC) for 32-bit integers, depending on the
underlying architecture.

\paragraph{Conversion.}
If $x$ is converted from integer type $s$ to $t$, the conversion
may lead to an integer error if the value of $x$ does not fall into
the range of $t$, e.g., converting a possibly negative signed integer
into unsigned.

\autoref{f:ax25-sign} shows such a conversion bug.  The
userspace-controlled signed integer \cc{optlen} is silently converted
to unsigned to be compared with \cc{sizeof(int)}.  The conversion
is regarded as an integer error since \cc{optlen} can go negative,
say -1, which will be misinterpreted as the positive value $2^{32}-1$
and bypass the sanity check $\cc{optlen} < \cc{sizeof(int)}$.


%in \autoref{f:ext4} the call to
%\cc{ext4_split_extent} returns a signed int, which will be negative
%on error.  However, the value is converted to an unsigned integer,
%leading to a meaningless comparison that is always false.

\begin{figure}
\centering
\begin{Verbatim}[commandchars=\\\{\},codes={\catcode`\$=3\catcode`\^=7\catcode`\_=8}]
\PY{c+cp}{\PYZsh{}}\PY{c+cp}{define IFNAMSIZ 16}
\PY{k}{static} \PY{k+kt}{int} \PY{n+nf}{ax25\PYZus{}setsockopt}\PY{p}{(}\PY{p}{.}\PY{p}{.}\PY{p}{.}\PY{p}{,}
    \PY{k+kt}{char} \PY{n}{\PYZus{}\PYZus{}user} \PY{o}{*}\PY{n}{optval}\PY{p}{,} \PY{k+kt}{int} \PY{n}{optlen}\PY{p}{)}
\PY{p}{\PYZob{}}
    \PY{k+kt}{char} \PY{n}{devname}\PY{p}{[}\PY{n}{IFNAMSIZ}\PY{p}{]}\PY{p}{;}
    \PY{p}{.}\PY{p}{.}\PY{p}{.}
    \PY{c+cm}{/* assume optlen = 0xffffffff */}
    \PY{c+cm}{/* optlen is treated as unsigned: $2^{32}-1$ */}
    \PY{k}{if} \PY{p}{(}\PY{n}{optlen} \PY{o}{<} \PY{k}{sizeof}\PY{p}{(}\PY{k+kt}{int}\PY{p}{)}\PY{p}{)}
        \PY{k}{return} \PY{o}{-}\PY{n}{EINVAL}\PY{p}{;}
    \PY{c+cm}{/* optlen is treated as signed: $-1$ */}
    \PY{k}{if} \PY{p}{(}\PY{n}{optlen} \PY{o}{>} \PY{n}{IFNAMSIZ}\PY{p}{)}
        \PY{n}{optlen} \PY{o}{=} \PY{n}{IFNAMSIZ}\PY{p}{;}
    \PY{n}{copy\PYZus{}from\PYZus{}user}\PY{p}{(}\PY{n}{devname}\PY{p}{,} \PY{n}{optval}\PY{p}{,} \PY{n}{optlen}\PY{p}{)}\PY{p}{;}
    \PY{p}{.}\PY{p}{.}\PY{p}{.}
\PY{p}{\PYZcb{}}
\end{Verbatim}

\vspace{-1em}
\caption{A signedness bug~\cite[CVE-2009-2909]{cve} in the AX.25
network protocol implementation of the Linux kernel.  A negative
\cc{optlen} will bypass the sanity check \cc{optlen < sizeof(int)}
and trigger an kernel oops in the subsequent \cc{copy_from_user}
call.}
\label{f:ax25-sign}
\end{figure}
\if 0
\begin{figure}
\centering
\begin{Verbatim}[commandchars=\\\{\}]
\PY{k}{static} \PY{k+kt}{int} \PY{n}{ext4\PYZus{}split\PYZus{}extent}\PY{p}{(}\PY{p}{.}\PY{p}{.}\PY{p}{.}\PY{p}{)}\PY{p}{;}

\PY{k}{static} \PY{k+kt}{int} \PY{n+nf}{ext4\PYZus{}ext\PYZus{}convert\PYZus{}to\PYZus{}initialized}\PY{p}{(}\PY{p}{.}\PY{p}{.}\PY{p}{.}\PY{p}{)}
\PY{p}{\PYZob{}}
	\PY{k+kt}{unsigned} \PY{k+kt}{int} \PY{n}{allocated}\PY{p}{;}
	\PY{k+kt}{int} \PY{n}{err} \PY{o}{=} \PY{l+m+mi}{0}\PY{p}{;}
	\PY{p}{.}\PY{p}{.}\PY{p}{.}
	\PY{n}{allocated} \PY{o}{=} \PY{n}{ext4\PYZus{}split\PYZus{}extent}\PY{p}{(}\PY{p}{.}\PY{p}{.}\PY{p}{.}\PY{p}{)}\PY{p}{;}
	\PY{k}{if} \PY{p}{(}\PY{n}{allocated} \PY{o}{<} \PY{l+m+mi}{0}\PY{p}{)}
		\PY{n}{err} \PY{o}{=} \PY{n}{allocated}\PY{p}{;}
	\PY{k}{return} \PY{n}{err} \PY{o}{?} \PY{n}{err} \PY{o}{:} \PY{n}{allocated}\PY{p}{;}
\PY{p}{\PYZcb{}}
\end{Verbatim}

\vspace{-1em}
\caption{A signedness bug in the ext4 filesystem of the Linux kernel.
Since \cc{allocated} is declared as unsigned int, the test
$(\cc{allocated} < 0)$ will always be false, which breaks the
error handling logic.}
\label{f:ext4}
\end{figure}
\fi

\subsection{A Strawman Analysis}

Consider a \naive analysis that generates constraints from both the
error and path preconditions.  As for the multiplication $\cc{maxnum}
\times_u 16$ in \autoref{f:bridge}, \sys computes its error
precondition as:
\begin{equation*}
\cc{maxnum} >_u (2^{32} - 1) / 16.
\end{equation*}
Since the multiplication is always reachable without any branches
in that function, the corresponding path precondition is simply true.

\sys then feeds the logical AND of the two preconditions into a
constraint solver~\cite{boolector}.  The solver computes a possible
input, e.g., $\cc{maxnum} = \cc{0xf0000000}$.

Now let's consider the patched code that correctly limits \cc{maxnum}
to 256, shown as below (\cc{maxnum} is
numbered~\cite[\chapterautorefname~8.11]{whale} for clarification
purpose).
\begin{Verbatim}[commandchars=\\\{\}]
\PY{n}{maxnum0} \PY{o}{=} \PY{p}{.}\PY{p}{.}\PY{p}{.}\PY{p}{;} \PY{c+cm}{/* read from userspace */} 
\PY{k}{if} \PY{p}{(}\PY{n}{maxnum0} \PY{o}{>} \PY{n}{PAGE\PYZus{}SIZE} \PY{o}{/} \PY{k}{sizeof}\PY{p}{(}\PY{k}{struct} \PY{n}{\PYZus{}\PYZus{}fdb\PYZus{}entry}\PY{p}{)}\PY{p}{)}
    \PY{n}{maxnum1} \PY{o}{=} \PY{n}{PAGE\PYZus{}SIZE} \PY{o}{/} \PY{k}{sizeof}\PY{p}{(}\PY{k}{struct} \PY{n}{\PYZus{}\PYZus{}fdb\PYZus{}entry}\PY{p}{)}\PY{p}{;}
\PY{k}{else}
    \PY{n}{maxnum1} \PY{o}{=} \PY{n}{maxnum}\PY{p}{;}
\PY{n}{size} \PY{o}{=} \PY{n}{maxnum1} \PY{o}{*} \PY{k}{sizeof}\PY{p}{(}\PY{k}{struct} \PY{n}{\PYZus{}\PYZus{}fdb\PYZus{}entry}\PY{p}{)}\PY{p}{;}
\PY{n}{buf} \PY{o}{=} \PY{n}{kmalloc}\PY{p}{(}\PY{n}{size}\PY{p}{,} \PY{n}{GFP\PYZus{}USER}\PY{p}{)}\PY{p}{;}
\PY{p}{.}\PY{p}{.}\PY{p}{.}
\end{Verbatim}

The error precondition of the multiplication remains unchanged.
\begin{equation*}
\cc{maxnum}_1 >_u (2^{32} - 1) / 16.
\end{equation*}
The corresponding path precondition is that \cc{maxnum} is reset to 256
if it is larger than 256, or remains the old value otherwise.
\begin{align*}
& ((\cc{maxnum}_0 >_u 256) \land (\cc{maxnum}_1 = 256)) \\
\lor
& (\neg (\cc{maxnum}_0 >_u 256) \land (\cc{maxnum}_1 = \cc{maxnum}_0)).
\end{align*}
Again \sys takes the logical AND of the two preconditions to the
constraint solver, which will conclude that these constraints can
never be satisfied.  This means that the integer error has been
fixed.

\subsection{Implication}
\label{s:imply}

With the guidelines for integer operations described in
\autoref{s:precondition}, a possibly counter-intuitive implication
is that two functionally equivalent operations may have different
error preconditions.  For example, $\cc{maxnum}\times_u 16$ may be
erroneous if \cc{maxnum} is large, while $\cc{maxnum} \shl 4$ is
considered always secure as long as \cc{maxinum} is wider than
4 bits.  The guidelines assume that the developer has chosen the
appropriate integer operation to match their intention.

Therefore, any compiler optimization that performs code transformations
like rewriting $\cc{maxnum}\times_u 16$ to $\cc{maxnum} \shl 4$
would destroy the precondition semantics.  Even worse, decent C
compilers may completely optimize away checks like $(x + 1) < x$
for signed integer $x$~\cite{gcc:signed-overflow,us-cert:gcc},
unless a special compile option \cc{-fwrapv} or \cc{-fno-strict-overflow}
is given.  To generate error constraints that best match the
developer's intention, \sys runs before these optimizations.

\if 0

\subsection{Challenges}
\label{s:chal}

There are several challenges that face \sys when applying the
secure integer standard described in \autoref{s:goal} to real-world
systems code.

\subsubsection{Benign Integer Errors}

While violating the secure integer standard, some commonly-used C
idioms will not cause any defects.  We recognize them as follows.

\paragraph{Partial violation.}
Take $(x +_u 1) -_u 2$ with $x \geq_u 1$ for an example.  The
expression would be considered as an integer error since the first
part $(x +_u 1)$ may be insecure, though the whole expression is
equivalent to $x -_u 1$ and will not cause any integer error.
Another example is that signed and unsigned integers are often used
interchangeably in C code.  In a conversion like \cc{(int)((unsigned)x)}
for a signed $x <_s 0$, the part \cc{(unsigned)x} may violate the
secure integer standard while the whole expression does not.  \sys
should avoid to warn against such ``partial'' violations.

\paragraph{Error-before-use.}
An integer error check may come after the overflowed computation,
but before any use of the result.  In that case, the overflowed
computation is benign.  Below is such an example.
\begin{Verbatim}[commandchars=\\\{\},codes={\catcode`\$=3\catcode`\^=7\catcode`\_=8}]
\PY{k+kt}{unsigned} \PY{n}{size} \PY{o}{=} \PY{n}{x} \PY{o}{*} \PY{n}{y}\PY{p}{;}
\PY{k}{if} \PY{p}{(}\PY{n}{x} \PY{o}{\PYZgt{}} \PY{n}{UINT\PYZus{}MAX} \PY{o}{/} \PY{n}{y}\PY{p}{)}
    \PY{k}{return} \PY{o}{-}\PY{l+m+mi}{1}\PY{p}{;}
\PY{p}{.}\PY{p}{.}\PY{p}{.} \PY{o}{=} \PY{n}{malloc}\PY{p}{(}\PY{n}{size}\PY{p}{)}\PY{p}{;}
\end{Verbatim}

Even the multiplication $x \times_u y$ overflows, the product
\cc{size} is not used before the check.  \sys will move the integer
operation $x \times_u y$ down to the latest possible point, i.e.,
right after the \cc{if} branch and before the \cc{malloc} call, so
as to avoid warning against the multiplication.

\paragraph{Overflowed checking idiom.}
It is commonly seen in practice to use an overflowed result to do
the integer error check for $x +_u y$:
\begin{align}
x +_u y <_u x.
\end{align}
This is equivalent to a ``sane'' check
$\cc{UINT_MAX} - x >_u y$.
\sys should recognize such integer error checking idioms and avoid
to warn against them.

Note that using overflowed result to check multiplication is trickier.
In general $x \times_u y <_u x$ is not a valid integer error check
but a bug.  A correct way would be $(x \times_u y) /_u y \neq x$
or a sane check, $\cc{UINT_MAX} /_u x > y$.

\subsubsection{Constraint Solving Performance}

Although \sys uses a highly-optimized constraint solver,
constraints generated unwisely would still hurt its performance,
sometimes even making it run forever.

\paragraph{Bounded constraint size.}
It is not a good idea to naively analyze and generate constraints
interprocedurally, for example,  across the whole Linux kernel.
The size of the path constraint would grow exponentially, which is
unnecessary and hard to solve.  To achieve scalability, \sys should
choose an appropriate program granularity.

\sys also needs to handle complex program constructs such as loops
and pointer arithmetic appropriately.  The generated constraints
should be able to catch common integer errors while being solvable
in a reasonable amount of time.

\paragraph{Idioms for faster solving.}
We notice that some operations like division would significantly
slow down the constraint solver~\cite{brummayer:perf}, most of which
are used in integer error checks like $\cc{UINT_MAX} /_u x > y$.
\sys should recognize these idioms and generate constraints that
are easier to solve.

\fi
