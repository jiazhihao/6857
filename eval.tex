\section{Evaluation}
\label{s:eval}

The evaluation tries to answer the following questions.
\begin{itemize}
\item
Is \sys effective to catch integer errors in buggy code?
\item
Will \sys declare that patched code does not contain no integer errors?
\item
Can \sys find integer errors in real-world systems?
\end{itemize}

\subsection{Validation}

We have collected XXX integer errors (both original and patched
code snippets) in the Linux kernel from the CVE list~\cite{cve} in
recent three years and use them as the benchmark to validate \sys.

\newcommand{\ok}{\textcolor{JungleGreen}{\checkmark}\xspace}
\newcommand{\checked}{$\boxtimes$}
\begin{figure*}
\centering
\footnotesize
\begin{tabular}{lllll} \toprule
 & Subsystem & Error & Caught in original? & Cleared in patch? \\ \midrule
\cc{arch:x86} \\
\hspace{1em} CVE-2009-3638 & kvm
 & $\times_u$, size & \ok & \ok \\
\cc{drivers:block} \\
\hspace{1em} CVE-2010-3437 & pktcdvd
 & index & \ok & \ok \\
\cc{drivers:char} \\
\hspace{1em} CVE-2011-2022 & agp
 & $+_u$ & \ok & \ok \\
\hspace{1em} CVE-2011-1746 & agp
 & $+_u$, $\times_u$ & \ok & \ok \\
\hspace{1em} CVE-2011-1745 & agp
 & $+_u$ & \ok & \ok \\
\cc{drivers:drm} \\
\hspace{1em} CVE-2011-1013 & irq
 & index & \ok & \ok \\
\cc{drivers:infiniband} \\
\hspace{1em} CVE-2010-4649 & uverbs
 & $\times_u$, size & \ok & \ok \\
\cc{drivers:media} \\
\hspace{1em} CVE-2011-0521 & av7110
 & index & \ok & \ok \\
\cc{drivers:net} \\
\hspace{1em} CVE-2009-1385 & e1000
 & $-_u$ & \ok & \ok \\
\cc{drivers:scsi} \\
\hspace{1em} CVE-2011-1494 & mpt2sas
 & $\times_u$, size & \ok & \ok \\
\hspace{1em} CVE-2010-4157 & gdth
 & $+_u$, size & \ok & \ok \\
\cc{fs} \\
\hspace{1em} CVE-2011-4077 & xfs
 & index, size & \ok & \ok \\
\hspace{1em} CVE-2011-3191 & cifs
 & $+_u$, index, size & \ok & \ok \\
\hspace{1em} CVE-2010-4162 & bio
 & $+_u$, $-_u$ & \ok & \ok \\
\hspace{1em} CVE-2010-3067 & aio
 & $\times_u$ & \ok & \ok \\
\hspace{1em} CVE-2010-2538 & btrfs
 & $+_u$, $-_u$ & \ok & solver timeout \\
\hspace{1em} CVE-2009-4307 & ext4
 & $\shl$, $/_u$ (ppc) & \shl semantics & undefined behavior \\
\cc{kernel} \\
\hspace{1em} CVE-2011-1593 & pid
 & $+_s$, index & \ok & \ok \\
\cc{mm} \\
\hspace{1em} CVE-2011-4097 & oom
 & $+_s$, $-_s$, $\times_s$ & \ok & page semantics \\
\cc{net} \\
\hspace{1em} CVE-2011-2497 & l2cap
 & $-_u$, size & \ok & \ok \\
\hspace{1em} CVE-2011-1770 & dccp
 & $-_u$ & \ok & \ok \\
\hspace{1em} CVE-2010-4529 & irda
 & $+_u$, $\times_s$, size & \ok & new errors \\
\hspace{1em} CVE-2010-4175 & rds
 & size & \ok & \ok \\
\hspace{1em} CVE-2010-4165 & tcp
 & $/_u$, $\times_u$ & \ok & \ok \\
\hspace{1em} CVE-2010-4164 & x25
 & $-_u$ & \ok & \ok \\
\hspace{1em} CVE-2010-3873 & x25
 & $-_u$, size & \ok & CVE-2010-4164 \\
\hspace{1em} CVE-2010-3865 & rds
 & $+_u$ & accumulation & \ok \\
\hspace{1em} CVE-2010-3310 & rose
 & index & \ok & \ok \\
\hspace{1em} CVE-2010-2959 & can
 & $+_u$, $\times_u$, size & \ok & \ok \\
\hspace{1em} CVE-2010-2478 & ethtool
 & $\times_u$, size & \ok & \ok \\
\hspace{1em} CVE-2009-3280 & cfg80211
 & $-_u$   & \ok & \ok \\
\hspace{1em} CVE-2009-2909 & ax25
 & $\times_s$, size & \ok & new errors \\
\hspace{1em} CVE-2009-1265 & rose (netrom, x25)
 & $+_u$, size & \ok & \ok \\
\hspace{1em} CVE-2008-3526 & sctp
 & $+_u$ & \ok & bad fix \\
\cc{sound} \\
\hspace{1em} CVE-2011-1477 & opl3
 & index & \ok & \ok \\
\hspace{1em} CVE-2010-3442 & alsa
 & $+_u$, $\times_u$ & \ok & \ok \\
\bottomrule
\end{tabular}

\caption{The result of applying \sys to integer errors in Linux
kernel from the CVE list.}
\end{figure*}

\subsubsection{False Negatives}

\sys does not catch two integer errors, both of which have been
discussed in \autoref{s:gen:limit}.  We detail them below.

\xw{show code?}

\paragraph{CVE-2009-4307.}
\sys is able to detect the oversized shift, but does not catch the
division by zero, which could happen on PowerPC but not x86, since
\sys's default semantics does not model the PowerPC architecture.

\paragraph{CVE-2010-3865.}
The addition overflow happens in a loop, which is an accumulation.
\sys cannot catch the bug since it unrolls the loop only once.

\subsubsection{False Positives and New Bugs}


\subsection{Case Study}

We periodically run \sys against the source code of the mainline
Linux kernel and submit patches to fix integer errors \sys reports.

\autoref{f:data:linux} summarizes the results.  The names for most
columns are self-explanatory.  ``Description'' shows the attack
vector and what values are infected.  ``\# of prev. fixes'' shows
the number of previous commits that tried to fix the same error but
didn't do it correctly.  ``Patch accepted?'' shows how the kernel
developers respond to our patches, as follows.
%We additional specify which repository the patch goes
%into, e.g., (net-next), if that patch has not reached the mainline.
\begin{itemize}
\item
\ok: the kernel developers accepted the patch upstream.
\item
Modified: the kernel developers modified our patch for inclusion.
\item
Acked: the kernel developers acknowledged the bug,
but we cannot find the patch showing up in any repository.
\item
Fixed: we withdrew the patch because the bug discovered in the
mainline was already fixed in some other repository.
\end{itemize}
We omit patches that does not get further responses. [[[numbers?]]]

\begin{figure*}
\centering
\footnotesize
\begin{tabular}{llllll} \toprule
Component & Error & Impact & Description & \# of prev. fixes & Patch accepted? \\ \midrule
\cc{drivers:drm} \\
\hspace{1em} crtc
	& $\times_u$
	& OOB write
	& ioctl: \cc{kmalloc} size
	& --
	& \ok \\
\hspace{1em} vmwgfx
	& $\times_u$
	& OOB read
	& ioctl: \cc{kmalloc} size
	& --
	& \ok \\
%\hspace{1em} vmwgfx
	& $\times_u$
	& OOB write
	& ioctl: VRAM size requested
	& 2
	& \ok \\
\cc{drivers:input} \\
\hspace{1em} cma3000_d0x
	& cmp
	& logic error
	& $\cc{u8} <_u 0$: error handling
	& --
	& \ok \\
\cc{drivers:media} \\
\hspace{1em} uvc
	& $\times_u$
	& OOB read
	& ioctl: \cc{kmalloc} size
	& --
	& \ok (uvcvideo) \\
\hspace{1em} wl128x
	& cmp (36)
	& logic error
	& $\cc{u32} <_u 0$: error handling
	& --
	& \ok (media_tree) \\
\cc{drivers:staging} \\
\hspace{1em} comedi
	& $\times_u$
	& OOB write
	& ioctl: \cc{kmalloc} size
	& 1
	& \ok \\
\hspace{1em} olpc_dcon
	& cmp
	& logic error
	& $\cc{(int)u8} = -1$: error handling
	& --
	& \ok (staging) \\
\hspace{1em} vt6655
	& $\times_u$ (2)
	& OOB write
	& ioctl: \cc{kmalloc} size
	& --
	& \ok (staging) \\
\hspace{1em} vt6656
	& $\times_u$ (2)
	& OOB write
	& ioctl: \cc{kmalloc} size
	& --
	& \ok (staging) \\
\cc{fs} \\
\hspace{1em} ext4
	& cmp
	& logic error
	& $\cc{uint} <_u 0$: error handling
	& --
	& fixed by others \\
\hspace{1em} nifs2
	& $\times_u$
	& OOB read
	& ioctl: \cc{vmalloc} size
	& --
	& \ok \\
\hspace{1em} xfs
	& $\times_u$
	& OOB write
	& disk: \cc{kmalloc} size
	& 1
	& \ok (xfs) \\
\cc{mm} \\
\hspace{1em} vmscan
	& cmp
	& logic error
	& $\cc{ulong} <_u 0$: overflow check
	& --
	& fixed by others \\
\cc{net} \\
\hspace{1em} ax25
	& $\times_u$ (8)
	& timer
	& {ioctl}/{setsockopt}: T1/T2/T3/idle timers
	& --
	& \ok modified (net-next) \\
%\hspace{1em} ax25
	& $\times_u$ (4)
	& timer
	& {setsockopt}: T1/T2/T3/idle timers
	& 1
	& \ok (net-next) \\
\hspace{1em} irda
	& $\times_s$
	& timer
	& {getsockopt}: watchdog timer
	& --
	& \ok (net-next) \\
\hspace{1em} netfilter
	& $-_u$ (2)
	& output
	& wrong timeout value sent out
	& --
	& \ok \\
\hspace{1em} netrom
	& $\times_u$ (4)
	& timer
	& {setsockopt}: T1/T2/T4/idle timers
	& --
	& \ok (net-next) \\
\hspace{1em} rps
	& $\times_u$
	& OOB write
	& sysfs: \cc{vmalloc} size
	& --
	& \ok \\
\hspace{1em} sctp
	& $\times_u$
	& timer
	& {setsockopt}: autoclose timer
	& 3
	& \ok \\
%\hspace{1em} sctp
	& N/A
	& N/A
	& CVE-2008-3526 not exploitable
	& 1
	& \ok \\
\bottomrule
\end{tabular}


\caption{Integer errors found in the latest Linux kernel by \sys.}
\label{f:data:linux}
\end{figure*}

User-space applications.

Currently \sys's annotations and ranking heuristics are tailored
for the Linux kernel.  We only run \sys against two popular user-space
applications.  The developers have promptly fixed 5 integer errors
reported by \sys: 1 for lighttpd and 4 for OpenSSH.

\subsection{Patching}

What do patches look like?

simple sanity check.
- return -EINVAL or silently limit the value (e.g., timeout).

interface change (e.g., CVE-2009-2909)
sometimes infeasible if the interface is exposed to userspace.

misread specification (CAN bits).
