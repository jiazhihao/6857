\section{Overview}
\label{s:oview}

This section gives an overview of \sys's goals and challenges.

\subsection{Integer Security}

\sys adopts the notion of integer security from the CERT C secure
coding standard~\cite[Ch.~5]{seacord:secure-c}.  The goal of \sys
is to warn insecure integer operations that violate the coding standard.
We summarize them as follows (see \autoref{s:sema} for a formal
definition).

\sys assumes two's complement~\cite[Ch.~4.2.1]{intel:vol1}, a de
facto standard integer representation on modern architectures.
$n$ bits are used to represent either unsigned integers in the range
$0$ to $2^n-1$, or signed integers in the range $-2^{n-1}$ to
$2^{n-1}-1$, with the most significant bit indicating the sign.  If
not particularly indicated, $x$ and $y$ are $n$-bit integers in
this subsection.

\noindent
{\it Addition \& subtraction}.
Perform the addition / subtraction as $(n+1)$-bit integers.
The operation is insecure if the result does not fall into the range of
$n$-bit integers.  For example, ...

\noindent
{\it Multiplication}.
Perform the multiplication as $2n$-bit integers.
The operation is insecure if the result does not fall
into the range of $n$-bit integers.

\noindent
{\it Division}.
The operation is insecure if the divisor is 0, Additionally, signed
division $x\ /_s\ y$ is insecure if $x = -2^{n-1}$ and $y = -1$,
because the result $2^{n-1}$ cannot be represented by an $n$-bit
signed integer.

\noindent
{\it Shift}. $x \shl y$ and $x \shr y$ are insecure if $y \geq_u n$,
because oversized shifts are undefined in C.  For example, $(1 \shl
32)$ may yield 0 or 1 for 32-bit integers, depending on the underlying
architecture.

\noindent
{\it Conversion}.
If $x$ is converted from type $s$ to $t$, the conversion is insecure
if the value of $x$ does not fall into the range of $t$.  For example,
...

\subsection{A Strawman Approach}

\sys first performs symbolic analysis to compute the insecure condition
and the corresponding path condition for each integer operation.
Take $\cc{arg->num_outputs} \times_u 16$ in
\autoref{f:vmwgfx} for an example.
From \autoref{s:sema} we can compute its insecure condition is:
\begin{equation*}
\cc{arg->num_outputs} >_u (2^{32} - 1) / 16.
\end{equation*}
To reach the multiplication, the corresponding path condition is
that the previous branching \cc{if (!arg->num_outputs)} fails:
\begin{equation*}
\cc{arg->num_outputs} \neq 0.
\end{equation*}

\sys then feeds the logical AND of the two conditions into a
constraint solver~\cite{boolector}.  The solver gives a solution

consider two patches.

a bad one.
$x * y < x$.

and a correct one.

\subsection{Challenges}

NB: equivalence.
$x \times 16$ is not the same as $x \shl 4$.

type conversion: ppl mix signed/unsigned integers.

pointer arithmetic.

overflow checking idioms.

loops.

solver performance.
