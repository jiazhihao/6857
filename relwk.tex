\section{Related Work}
\label{s:relwk}

This section relates \sys to previous work.

A simple static analysis technique is to extend the type system to
distinguish trusted and untrusted integers, and then perform taint
analysis to detect untrusted integers used in sensitive sinks~\cite{cqual,
lclint}.  Since path conditions (e.g., sanity checks) are not
tracked, such tools would report false errors on all correctly fixed
code.

The range checker from Stanford's metacompiler~\cite{range-checker}
eliminates cases where a user-controlled value is checked against
\emph{some} bound and reports unbounded values.  A similar heuristic
is used in a PREfast-based tool~\cite{prefast} inside Microsoft.
This approach will miss integer errors due to incorrect bounds.

PREfix~\cite{moy:prefix}

Carburizer~\cite{kadav:tolerating}.
\sys trusts hardware devices and filters out integer errors
caused by hardware.  One could disable the filter in \sys
to find such bugs for improving reliability of device code.


compiler

RICH~\cite{brumley:rich}

IOC

Pex

SAGE

ESBMC, KLEE

Linux fuzzing testing tools
like fsfuzzer and Trinity.
