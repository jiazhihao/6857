\section{Common Pitfalls}
\label{s:common}

This section summarizes common integer error patterns and the way
to fix them.

A large portioin of integer errors found in the Linux kernel are
due to missing or incorrect bound checks for data from untrusted
sources.  The patch is simply to add or correct the sanity check.


char$\to$int: 
CA-1996-22,
Apache,
lighttpd new bug

bad overflow checks: $x \times y < x$, 
\autoref{f:bridge}.

compilers and arch:
signed overflow (IntegerLib, etc.),
oversized shift (Google's Native Client sandboxing),
ABI on 64-bit S/390, PowerPC, SPARC, and MIPS
and 32-bit system call parameters~\cite[CVE-2009-0029]{cve}

pivot in binary search.

index checks:
miss (i < 0) or should declare i as unsigned

return values: (ret < 0)
should declare ret as signed.

What do patches look like?

simple sanity check.
- return -EINVAL or silently limit the value (e.g., timeout).

interface change (e.g., CVE-2009-2909)
sometimes infeasible if the interface is exposed to userspace.


malloc $\to$ calloc

choose the right type (and sign).

use the correct check.

type promotion.

separate error code with bad values

when writing your own allocator, don't repeat the mistake that
happened in glibc and Microsoft's libc~\cite{rus-cert:calloc}.
