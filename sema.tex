\section{Integer Semantics}
\label{s:sema}

We define \sys's integer semantics via the security requirement of
each integer operation.  Any integer operation that violates the
security requirement implies an error, unless the operation is in
the white list of code patterns that \sys explicitly allows.

\subsection{Secure Integer Operations}
\label{s:sema:sec}

\sys assumes two's complement~\cite[\chapterautorefname~4.2.1]{intel:vol1},
a de facto standard integer representation on modern architectures.
An $n$-bit unsigned integer is in the range $0$ to $2^n-1$, while
an $n$-bit signed integer is in the range $-2^{n-1}$ to $2^{n-1}-1$,
with the most significant bit indicating the sign.  An operation
may have a subscript $s$ or $u$ to indicate whether it operates on
signed or unsigned integers, respectively.

\paragraph{Addition \& subtraction \& multiplication.}
The result of an additive or multiplicative operation should fall
in the predefined range.  For example,
$\cc{maxnum}\times_u 16$ causes an integer error if $\cc{maxnum} =
\cc{0xf0000000}$, because the expected product $2^{32}\times
3\times 5$ is out of the range of $32$-bit unsigned integers~($0$
to $2^{32} - 1$).

\paragraph{Division.}
A division operation causes an integer error if the divisor is 0.
Additionally, signed division $-2^{n-1} /_s {-1}$ may lead to an integer
error, because the expected quotient $2^{n-1}$ is not in
the range of $n$-bit signed integers~($-2^{n-1}$ to $2^{n-1}-1$).

\paragraph{Shift.}
For $n$-bit integers, the shifting amount should be non-negative
and at most $n-1$.  Otherwise, the operation is undefined according
to the C standard.  As for 32-bit integers, $1 \shl 32$ yields 1
on x86 and 0 on PowerPC.  \sys considers the operation as an integer
error.

\paragraph{Conversion.}
Lossy truncations and sign conversions are often seen in C
code, and they can be found by invoking GCC with \cc{-Wconversion}.
\sys does not consider such a conversion as an integer error, unless
its result is used in one of the following two cases:

\paragraph{$(1)$ Tautological comparison.}
A comparison is always true or false due to an integer conversion,
which means that the result from the conversion voids one of the
branches.

For example, in \autoref{f:ext4} the call to \cc{ext4_split_extent}
returns a signed integer, which will be negative on error.  However,
the return value is converted to an unsigned integer for comparison,
leading to a stale expression $\cc{allocated} <_u 0$ that always
evaluates to false.

\begin{figure}
\centering
\begin{Verbatim}[commandchars=\\\{\}]
\PY{k}{static} \PY{k+kt}{int} \PY{n}{ext4\PYZus{}split\PYZus{}extent}\PY{p}{(}\PY{p}{.}\PY{p}{.}\PY{p}{.}\PY{p}{)}\PY{p}{;}

\PY{k}{static} \PY{k+kt}{int} \PY{n+nf}{ext4\PYZus{}ext\PYZus{}convert\PYZus{}to\PYZus{}initialized}\PY{p}{(}\PY{p}{.}\PY{p}{.}\PY{p}{.}\PY{p}{)}
\PY{p}{\PYZob{}}
	\PY{k+kt}{unsigned} \PY{k+kt}{int} \PY{n}{allocated}\PY{p}{;}
	\PY{k+kt}{int} \PY{n}{err} \PY{o}{=} \PY{l+m+mi}{0}\PY{p}{;}
	\PY{p}{.}\PY{p}{.}\PY{p}{.}
	\PY{n}{allocated} \PY{o}{=} \PY{n}{ext4\PYZus{}split\PYZus{}extent}\PY{p}{(}\PY{p}{.}\PY{p}{.}\PY{p}{.}\PY{p}{)}\PY{p}{;}
	\PY{k}{if} \PY{p}{(}\PY{n}{allocated} \PY{o}{<} \PY{l+m+mi}{0}\PY{p}{)}
		\PY{n}{err} \PY{o}{=} \PY{n}{allocated}\PY{p}{;}
	\PY{k}{return} \PY{n}{err} \PY{o}{?} \PY{n}{err} \PY{o}{:} \PY{n}{allocated}\PY{p}{;}
\PY{p}{\PYZcb{}}
\end{Verbatim}

\vspace{-1em}
\caption{An integer error in the ext4 filesystem of the Linux kernel.
Since \cc{allocated} is declared as \cc{unsigned int}, the test
$(\cc{allocated} < 0)$ will always be false, which breaks the
error handling logic.}
\label{f:ext4}
\end{figure}

\paragraph{$(2)$ Negative count.}
An integer that should always be non-negative is to be negative.
Such conceptually non-negative cases including array indices and
``size'' function parameters.

\autoref{f:ax25-sign} shows such an example in the Linux kernel.
The third parameter of $\cc{copy_from_user}(\cc{dst}, \cc{src},
\cc{size})$ should be non-negative since it indicates the number
of bytes to be copied from user space to kernel.  However, an
adversary could supply $-1$ for \cc{optlen} from user space, which
will bypass both sanity checks because the value is interpreted as
unsigned in the first comparison and signed in the second.  \cc{optlen}
is later used in the call to \cc{copy_from_user} as the size
parameter, which flags an integer error since \cc{optlen} is negative.

\begin{figure}
\centering
\begin{Verbatim}[commandchars=\\\{\},codes={\catcode`\$=3\catcode`\^=7\catcode`\_=8}]
\PY{c+cp}{\PYZsh{}}\PY{c+cp}{define IFNAMSIZ 16}
\PY{k}{static} \PY{k+kt}{int} \PY{n+nf}{ax25\PYZus{}setsockopt}\PY{p}{(}\PY{p}{.}\PY{p}{.}\PY{p}{.}\PY{p}{,}
    \PY{k+kt}{char} \PY{n}{\PYZus{}\PYZus{}user} \PY{o}{*}\PY{n}{optval}\PY{p}{,} \PY{k+kt}{int} \PY{n}{optlen}\PY{p}{)}
\PY{p}{\PYZob{}}
    \PY{k+kt}{char} \PY{n}{devname}\PY{p}{[}\PY{n}{IFNAMSIZ}\PY{p}{]}\PY{p}{;}
    \PY{p}{.}\PY{p}{.}\PY{p}{.}
    \PY{c+cm}{/* assume optlen = 0xffffffff */}
    \PY{c+cm}{/* optlen is treated as unsigned: $2^{32}-1$ */}
    \PY{k}{if} \PY{p}{(}\PY{n}{optlen} \PY{o}{<} \PY{k}{sizeof}\PY{p}{(}\PY{k+kt}{int}\PY{p}{)}\PY{p}{)}
        \PY{k}{return} \PY{o}{-}\PY{n}{EINVAL}\PY{p}{;}
    \PY{c+cm}{/* optlen is treated as signed: $-1$ */}
    \PY{k}{if} \PY{p}{(}\PY{n}{optlen} \PY{o}{>} \PY{n}{IFNAMSIZ}\PY{p}{)}
        \PY{n}{optlen} \PY{o}{=} \PY{n}{IFNAMSIZ}\PY{p}{;}
    \PY{n}{copy\PYZus{}from\PYZus{}user}\PY{p}{(}\PY{n}{devname}\PY{p}{,} \PY{n}{optval}\PY{p}{,} \PY{n}{optlen}\PY{p}{)}\PY{p}{;}
    \PY{p}{.}\PY{p}{.}\PY{p}{.}
\PY{p}{\PYZcb{}}
\end{Verbatim}

\vspace{-1em}
\caption{An integer error (CVE-2009-2909) in the AX.25
network protocol implementation of the Linux kernel.  A negative
\cc{optlen} will bypass both sanity checks due to sign misinterpretation
 and trigger an kernel
oops in the subsequent \cc{copy_from_user} call.}
\label{f:ax25-sign}
\end{figure}

\subsection{Equivalence}
\label{s:sema:eqv}

It is worth noting that two functionally equivalent operations at
the machine instruction level may have different integer semantics.
For example, $\cc{maxnum}\times_u 16$ and $\cc{maxnum} \shl 4$ are
identical instructions, but the former expression causes an integer
error if \cc{maxnum} is large, while the latter one is considered
always safe.  \sys assumes that the developer has chosen the
intended integer operation.

Any compiler optimization that rewrites
$\cc{maxnum}\times_u 16$ to $\cc{maxnum} \shl 4$ will destroy the
integer semantics.  Even worse, C compilers like GCC may completely
optimize away checks like $(x + 1) < x$ if $x$ is a signed integer
or a pointer~\cite{gcc:signed-overflow,us-cert:gcc}, unless a special
option \cc{-fwrapv} or \cc{-fno-strict-overflow} is
given.  To avoid the interference, \sys is designed to run before
these optimizations.

\subsection{Default Semantics}
\label{s:sema:def}

Signed integer overflows and oversized shifts are undefined according
to the C language standard.  In order to continue evaluating
expressions like $x \times_s y + z$ after $x \times_s y$ overflows,
\sys defines the default semantics for these integer operations, as
follows:
\begin{itemize}
\item
For $n$-bit signed arithmetic operations, \sys assumes the wrapping
semantics as if they were unsigned (i.e., $\mod{2^n}$), which is
the same as invoking GCC with the \cc{-fwrapv} option.
\item
Division by zero is undefined.  \sys does not consider the result
in further expressions.
\item
For shifts, \sys assumes x86's semantics, using $\log_2 n$ bits for
shifting an $n$-bit integer.  That is, it uses 5~bits of the shift
amount for 32-bit shifts, e.g., $1 \shl 32 = 1$.  This semantics
is also the behavior of ARM and MIPS processors.
\end{itemize}
\sys may miss integer errors that are not modeled by the above
semantics.  Consider the expression $x / (1 \shl y)$, where $x$ and
$y$ are unconstrained 32-bit integers.  \sys will be able to detect
the oversized shift via error constraint $y \geq_u 32$.  However,
it will miss the division-by-zero error, which is possible on PowerPC
but not on x86.

\subsection{White Listing Idioms}
\label{s:sema:whitelist}

It is common in practice to use an overflowed result to perform
an integer error check for $x +_u y$, where both $x$ and $y$ are
$n$-bit unsigned integers, as follows:
\begin{equation*}
x +_u y <_u x.
\end{equation*}
This idiom is useful when the sum is needed later in the code.
\sys recognizes such idioms listed in \autoref{f:whitelist} and
does not consider them as integer errors.

An equivalent non-overflow form of each overflowed check idiom is
also shown for comparison.  Let $\uintmax(n)$ denote the largest
$n$-bit unsigned integer, i.e., $2^n - 1$.  Below is a non-overflow
version of addition overflow check:
\begin{equation*}
\uintmax(n) -_u x >_u y.
\end{equation*}

Note that \sys does not recognize the overflowed comparison $x
\times_u y <_u x$ as a valid integer error check.  For example,
given $x = \cc{0x1fffffff}$ and $y = 16$, their product evaluates
to a larger value $\cc{0xfffffff0}$, which both overflows and
bypasses the check.  A correct check is $(x \times_u y) /_u y \neq
x$, or an non-overflow form, $\uintmax(n) /_u x > y$.

In addition, \sys accepts user-provided idioms tailored for the
code being analyzed.  For example, in the Linux kernel to set a
timer that expires after \cc{delay} ticks one often invokes
\cc{mod_timer(..., jiffies + delay)}, where \cc{jiffies} is the
number of ticks since the machine started.  \sys recognizes
this idiom and ignores the addition \cc{jiffies + delay} that may
theoretically overflow.

\begin{figure}
\centering
\begin{tabular}{ll}
\toprule
Overflowed check & Equivalent check \\ \midrule
$x + y <_u x$ & $x >_u \uintmax(n) - y$ \\
$x - y <_s 0$ & $x <_u y$ \\
$(x \times y) /_u y \neq x$ & $x >_u \uintmax(n) /_u y$   \\
\bottomrule
\end{tabular}
\caption{Examples of overflowed check idioms that \sys recognizes.
Here $\uintmax(n)$ denotes the largest $n$-bit unsigned integer.}
\label{f:whitelist}
\end{figure}
