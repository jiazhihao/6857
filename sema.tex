\section{Integer Semantics}
\label{s:sema}

We define \sys's integer semantics via error constraints.  The error
constraint of an integer operation is a Boolean predicate if satisfied
implies an integer error, unless the operation is in the white list
consisting of code patterns that \sys explicitly allows.

\subsection{Error Constraint}
\label{s:sema:constr}

\sys assumes two's complement~\cite[\chapterautorefname~4.2.1]{intel:vol1},
a de facto standard integer representation on modern architectures.
An $n$-bit unsigned integer is in the range $0$ to $2^n-1$, while
an $n$-bit signed integer is in the range $-2^{n-1}$ to $2^{n-1}-1$,
with the most significant bit indicating the sign.  A subscript $s$
or $u$ may be pinned to an operation to indicate it operates on
signed or unsigned integers, respectively.

\if 0
\sys adopts the CERT C secure coding guidelines for integer
operations~\cite[\chapterautorefname~5]{seacord:secure-c} as a basis
and considers violations of the guidelines as integer errors.  These
violations cover a set of integer-based vulnerabilities, sometimes
referred as arithmetic overflow (underflow, wraparound), division-by-zero,
oversized shift, and signedness bugs in the literature.  We summarize
them below.
\fi

\paragraph{Addition \& subtraction \& multiplication.}
An additive or multiplicative operation may lead to an integer error
if the result of the corresponding infinitely ranged mathematical
operation falls out of the predefined range.  For example,
$\cc{maxnum}\times_u 16$ causes an integer error if $\cc{maxnum} =
\cc{0xf0000000}$, because the mathematical product $\cc{0xf0000000}
\times 16 = 2^{32}\times 3\times 5$ is out of the range of $32$-bit
unsigned integers ($0$ to $2^{32} - 1$).

\paragraph{Division.}
A division operation causes an integer error if the divisor is 0.
Additionally, signed division $x\ /_s\ y$ may lead to an integer
error if $x = -2^{n-1}$ and $y = -1$, because the mathematical
result $2^{n-1}$ is not in the range of $n$-bit signed integers
($-2^{n-1}$ to $2^{n-1}-1$).

\paragraph{Shift.}
The bitwise shifts $x \shl y$ and $x \shr y$ are considered as
integer errors if $y \geq_u n$, which is undefined according to the
C standard.  For example, $(1 \shl 32)$ may yield 1 (e.g., on x86)
or 0 (e.g., on PowerPC) for 32-bit integers, depending on the
underlying architecture.

\paragraph{Conversion.}
Lossy truncations and sign conversions are often seen in systems
code, though one may find them by simply invoking GCC with
\cc{-Wconversion} for inspection.  In general \sys does not consider
such a conversion as an integer error, unless its result is used
in one of the following cases.
\begin{itemize}
\item
A comparison is made always true or false, which means that the
result from the conversion voids one of the branches.
\item
A theoretically non-negative value may become negative, including
array indices and ``size'' parameters, e.g., in a call to
\cc{copy_from_user}.
\end{itemize}

\autoref{f:ax25-sign} shows such an integer error in the Linux
kernel.  An adversary could supply $-1$ for \cc{optlen} from user
space, which will bypass both sanity checks because the value is
interpreted as unsigned in the first comparison and signed in the
second.  \cc{optlen} is later used in the call to \cc{copy_from_user}
as the size parameter, which flags an integer error since \cc{optlen}
is negative, i.e., its most significant bit is 1.

%in \autoref{f:ext4} the call to
%\cc{ext4_split_extent} returns a signed int, which will be negative
%on error.  However, the value is converted to an unsigned integer,
%leading to a meaningless comparison that is always false.

\begin{figure}
\centering
\begin{Verbatim}[commandchars=\\\{\},codes={\catcode`\$=3\catcode`\^=7\catcode`\_=8}]
\PY{c+cp}{\PYZsh{}}\PY{c+cp}{define IFNAMSIZ 16}
\PY{k}{static} \PY{k+kt}{int} \PY{n+nf}{ax25\PYZus{}setsockopt}\PY{p}{(}\PY{p}{.}\PY{p}{.}\PY{p}{.}\PY{p}{,}
    \PY{k+kt}{char} \PY{n}{\PYZus{}\PYZus{}user} \PY{o}{*}\PY{n}{optval}\PY{p}{,} \PY{k+kt}{int} \PY{n}{optlen}\PY{p}{)}
\PY{p}{\PYZob{}}
    \PY{k+kt}{char} \PY{n}{devname}\PY{p}{[}\PY{n}{IFNAMSIZ}\PY{p}{]}\PY{p}{;}
    \PY{p}{.}\PY{p}{.}\PY{p}{.}
    \PY{c+cm}{/* assume optlen = 0xffffffff */}
    \PY{c+cm}{/* optlen is treated as unsigned: $2^{32}-1$ */}
    \PY{k}{if} \PY{p}{(}\PY{n}{optlen} \PY{o}{<} \PY{k}{sizeof}\PY{p}{(}\PY{k+kt}{int}\PY{p}{)}\PY{p}{)}
        \PY{k}{return} \PY{o}{-}\PY{n}{EINVAL}\PY{p}{;}
    \PY{c+cm}{/* optlen is treated as signed: $-1$ */}
    \PY{k}{if} \PY{p}{(}\PY{n}{optlen} \PY{o}{>} \PY{n}{IFNAMSIZ}\PY{p}{)}
        \PY{n}{optlen} \PY{o}{=} \PY{n}{IFNAMSIZ}\PY{p}{;}
    \PY{n}{copy\PYZus{}from\PYZus{}user}\PY{p}{(}\PY{n}{devname}\PY{p}{,} \PY{n}{optval}\PY{p}{,} \PY{n}{optlen}\PY{p}{)}\PY{p}{;}
    \PY{p}{.}\PY{p}{.}\PY{p}{.}
\PY{p}{\PYZcb{}}
\end{Verbatim}

\vspace{-1em}
\caption{A signedness bug~\cite[CVE-2009-2909]{cve} in the AX.25
network protocol implementation of the Linux kernel.  A negative
\cc{optlen} will bypass both sanity checks due to sign misinterpretation
 and trigger an kernel
oops in the subsequent \cc{copy_from_user} call.}
\label{f:ax25-sign}
\end{figure}
\if 0
\begin{figure}
\centering
\begin{Verbatim}[commandchars=\\\{\}]
\PY{k}{static} \PY{k+kt}{int} \PY{n}{ext4\PYZus{}split\PYZus{}extent}\PY{p}{(}\PY{p}{.}\PY{p}{.}\PY{p}{.}\PY{p}{)}\PY{p}{;}

\PY{k}{static} \PY{k+kt}{int} \PY{n+nf}{ext4\PYZus{}ext\PYZus{}convert\PYZus{}to\PYZus{}initialized}\PY{p}{(}\PY{p}{.}\PY{p}{.}\PY{p}{.}\PY{p}{)}
\PY{p}{\PYZob{}}
	\PY{k+kt}{unsigned} \PY{k+kt}{int} \PY{n}{allocated}\PY{p}{;}
	\PY{k+kt}{int} \PY{n}{err} \PY{o}{=} \PY{l+m+mi}{0}\PY{p}{;}
	\PY{p}{.}\PY{p}{.}\PY{p}{.}
	\PY{n}{allocated} \PY{o}{=} \PY{n}{ext4\PYZus{}split\PYZus{}extent}\PY{p}{(}\PY{p}{.}\PY{p}{.}\PY{p}{.}\PY{p}{)}\PY{p}{;}
	\PY{k}{if} \PY{p}{(}\PY{n}{allocated} \PY{o}{<} \PY{l+m+mi}{0}\PY{p}{)}
		\PY{n}{err} \PY{o}{=} \PY{n}{allocated}\PY{p}{;}
	\PY{k}{return} \PY{n}{err} \PY{o}{?} \PY{n}{err} \PY{o}{:} \PY{n}{allocated}\PY{p}{;}
\PY{p}{\PYZcb{}}
\end{Verbatim}

\vspace{-1em}
\caption{A signedness bug in the ext4 filesystem of the Linux kernel.
Since \cc{allocated} is declared as unsigned int, the test
$(\cc{allocated} < 0)$ will always be false, which breaks the
error handling logic.}
\label{f:ext4}
\end{figure}
\fi

It is worth noting that two functionally equivalent operations from
the perspective of an optimizing compiler may have different integer
semantics.  For example, $\cc{maxnum}\times_u 16$ and $\cc{maxnum}
\shl 4$ are identical at machine instruction level, but the former
expression causes an integer error if \cc{maxnum} is large, while
the latter one is considered always safe.  \sys assumes that the
developer has chosen the appropriate integer operation to match
their intention.

Therefore, any compiler optimization that would rewrite
$\cc{maxnum}\times_u 16$ to $\cc{maxnum} \shl 4$ will destroy the
integer semantics.  Even worse, a decent C compiler may completely
optimize away checks like $(x + 1) < x$ if $x$ is a signed integer
or a pointer~\cite{gcc:signed-overflow,us-cert:gcc}, unless a special
compile option \cc{-fwrapv} or \cc{-fno-strict-overflow} is given.
To generate error constraints that best match the developer's
intention, \sys runs before these optimizations.

\subsection{White List}
\label{s:sema:whitelist}

It is commonly seen in practice to use an overflowed result to do
the integer error check for $x +_u y$, where both $x$ and $y$ are
$n$-bit unsigned integers, as follows:
\begin{equation*}
x +_u y <_u x.
\end{equation*}
This idiom is useful when the sum is needed later in the code.
\sys recognizes such idioms listed in \autoref{f:whitelist} and
does not consider them as integer errors.

The equivalent ``sane'' form for each overflowed check idiom is
also shown for comparison.  Let $\umax(n)$ denote the largest $n$-bit
unsigned integer, i.e., $2^n - 1$.  The overflowed addition check
is equivalent to the sane form:
\begin{equation*}
\umax(n) - x >_u y.
\end{equation*}

Note that using overflowed result to check multiplication is trickier.
In general $x \times_u y <_u x$ is not a valid integer error check.
For example, given $\cc{maxnum} = \cc{0x1fffffff}$, $\cc{maxnum}
\times_u 16$ gives a larger value $\cc{0xfffffff0}$, which both
overflows and bypasses the check.  A correct check is $(x \times_u
y) /_u y \neq x$ or a sane form, $\umax(n) /_u x > y$.

Whether the imperfect check will lead to a security vulnerability
depends on the context.  In the multiplication example above,
\cc{maxnum} has to be at least $2^{28}$ to overflow $\cc{maxnum}
\times 16$, and the product must be greater than or equal to that
to bypass the check $\cc{maxnum} \times_u 16 <_u \cc{maxnum}$.
Allocating $2^{28}$ bytes (i.e., 256~MB) is possible in a user-space
application, but unlikely to succeed in the Linux kernel with
\cc{kmalloc}, which imposes a relatively small
limit~\cite[\chapterautorefname~8]{ldd3}.  Nevertheless, the check
becomes dangerous where the allocation function used has a bigger
limit (e.g., \cc{vmalloc}), the multiplication is $\cc{maxnum}
\times_u 1024$ (i.e., allocating a smaller chunk of memory), or the
\cc{kmalloc} limit is raised in the future.

Another error-prone check is oversized shift, e.g., using $(1 \shl
n) = 0$ to detect a large $n$.  As mentioned in \autoref{s:sema:constr},
${1 \shl n}$ is undefined in C; its value will never be 0 on
architectures like x86, which fails the check.

Currently \sys does not put these dangerous overflowed checking
forms in the white list, even though they may not be exploitable.

In addition, \sys accepts user-provided idioms tailored for the
code being analyzed.  For example, in the Linux kernel to set a
timer that expires after \cc{delay} ticks one often invokes
\cc{mod_timer(..., jiffies + delay)}, where \cc{jiffies} is the
number of ticks since the machine started.  \sys recognizes
such idioms and ignores the addition \cc{jiffies + delay} that may
overflow.

\begin{figure}
\centering
\begin{tabular}{ll}
\toprule
Overflowed check & Equivalent sane check \\ \midrule
$x + y <_u x$ & $x >_u \umax(n) - y$ \\
$x - y <_s 0$ & $x <_u y$ \\
$(x \times y) /_u y \neq x$ & $x >_u \umax(n) /_u y$   \\
\bottomrule
\end{tabular}
\caption{Examples of overflowed check idioms that \sys recognizes.
Here $\umax(n)$ denotes the largest $n$-bit unsigned integer.}
\label{f:whitelist}
\end{figure}

\if 0

\subsection{A Strawman Analysis}

Consider a \naive analysis that generates constraints from both the
error and path preconditions.  As for the multiplication $\cc{maxnum}
\times_u 16$ in \autoref{f:bridge}, \sys computes its error
precondition as:
\begin{equation*}
\cc{maxnum} >_u (2^{32} - 1) / 16.
\end{equation*}
Since the multiplication is always reachable without any branches
in that function, the corresponding path precondition is simply true.

\sys then feeds the logical AND of the two preconditions into a
constraint solver~\cite{boolector}.  The solver computes a possible
input, e.g., $\cc{maxnum} = \cc{0xf0000000}$.

Now let's consider the patched code that correctly limits \cc{maxnum}
to 256, shown as below (\cc{maxnum} is
numbered~\cite[\chapterautorefname~8.11]{whale} for clarification
purpose).
\begin{Verbatim}[commandchars=\\\{\}]
\PY{n}{maxnum0} \PY{o}{=} \PY{p}{.}\PY{p}{.}\PY{p}{.}\PY{p}{;} \PY{c+cm}{/* read from userspace */} 
\PY{k}{if} \PY{p}{(}\PY{n}{maxnum0} \PY{o}{>} \PY{n}{PAGE\PYZus{}SIZE} \PY{o}{/} \PY{k}{sizeof}\PY{p}{(}\PY{k}{struct} \PY{n}{\PYZus{}\PYZus{}fdb\PYZus{}entry}\PY{p}{)}\PY{p}{)}
    \PY{n}{maxnum1} \PY{o}{=} \PY{n}{PAGE\PYZus{}SIZE} \PY{o}{/} \PY{k}{sizeof}\PY{p}{(}\PY{k}{struct} \PY{n}{\PYZus{}\PYZus{}fdb\PYZus{}entry}\PY{p}{)}\PY{p}{;}
\PY{k}{else}
    \PY{n}{maxnum1} \PY{o}{=} \PY{n}{maxnum}\PY{p}{;}
\PY{n}{size} \PY{o}{=} \PY{n}{maxnum1} \PY{o}{*} \PY{k}{sizeof}\PY{p}{(}\PY{k}{struct} \PY{n}{\PYZus{}\PYZus{}fdb\PYZus{}entry}\PY{p}{)}\PY{p}{;}
\PY{n}{buf} \PY{o}{=} \PY{n}{kmalloc}\PY{p}{(}\PY{n}{size}\PY{p}{,} \PY{n}{GFP\PYZus{}USER}\PY{p}{)}\PY{p}{;}
\PY{p}{.}\PY{p}{.}\PY{p}{.}
\end{Verbatim}

The error precondition of the multiplication remains unchanged.
\begin{equation*}
\cc{maxnum}_1 >_u (2^{32} - 1) / 16.
\end{equation*}
The corresponding path precondition is that \cc{maxnum} is reset to 256
if it is larger than 256, or remains the old value otherwise.
\begin{align*}
& ((\cc{maxnum}_0 >_u 256) \land (\cc{maxnum}_1 = 256)) \\
\lor
& (\neg (\cc{maxnum}_0 >_u 256) \land (\cc{maxnum}_1 = \cc{maxnum}_0)).
\end{align*}
Again \sys takes the logical AND of the two preconditions to the
constraint solver, which will conclude that these constraints can
never be satisfied.  This means that the integer error has been
fixed.

\subsection{Challenges}
\label{s:chal}

There are several challenges that face \sys when applying the
secure integer standard described in \autoref{s:goal} to real-world
systems code.

\subsubsection{Benign Integer Errors}

While violating the secure integer standard, some commonly-used C
idioms will not cause any defects.  We recognize them as follows.

\paragraph{Partial violation.}
Take $(x +_u 1) -_u 2$ with $x \geq_u 1$ for an example.  The
expression would be considered as an integer error since the first
part $(x +_u 1)$ may be insecure, though the whole expression is
equivalent to $x -_u 1$ and will not cause any integer error.
Another example is that signed and unsigned integers are often used
interchangeably in C code.  In a conversion like \cc{(int)((unsigned)x)}
for a signed $x <_s 0$, the part \cc{(unsigned)x} may violate the
secure integer standard while the whole expression does not.  \sys
should avoid to warn against such ``partial'' violations.

\paragraph{Error-before-use.}
An integer error check may come after the overflowed computation,
but before any use of the result.  In that case, the overflowed
computation is benign.  Below is such an example.
\begin{Verbatim}[commandchars=\\\{\},codes={\catcode`\$=3\catcode`\^=7\catcode`\_=8}]
\PY{k+kt}{unsigned} \PY{n}{size} \PY{o}{=} \PY{n}{x} \PY{o}{*} \PY{n}{y}\PY{p}{;}
\PY{k}{if} \PY{p}{(}\PY{n}{x} \PY{o}{\PYZgt{}} \PY{n}{UINT\PYZus{}MAX} \PY{o}{/} \PY{n}{y}\PY{p}{)}
    \PY{k}{return} \PY{o}{-}\PY{l+m+mi}{1}\PY{p}{;}
\PY{p}{.}\PY{p}{.}\PY{p}{.} \PY{o}{=} \PY{n}{malloc}\PY{p}{(}\PY{n}{size}\PY{p}{)}\PY{p}{;}
\end{Verbatim}

Even the multiplication $x \times_u y$ overflows, the product
\cc{size} is not used before the check.  \sys will move the integer
operation $x \times_u y$ down to the latest possible point, i.e.,
right after the \cc{if} branch and before the \cc{malloc} call, so
as to avoid warning against the multiplication.

\paragraph{Overflowed checking idiom.}
It is commonly seen in practice to use an overflowed result to do
the integer error check for $x +_u y$:
\begin{align}
x +_u y <_u x.
\end{align}
This is equivalent to a ``sane'' check
$\cc{UINT_MAX} - x >_u y$.
\sys should recognize such integer error checking idioms and avoid
to warn against them.

Note that using overflowed result to check multiplication is trickier.
In general $x \times_u y <_u x$ is not a valid integer error check
but a bug.  A correct way would be $(x \times_u y) /_u y \neq x$
or a sane check, $\cc{UINT_MAX} /_u x > y$.

\subsubsection{Constraint Solving Performance}

Although \sys uses a highly-optimized constraint solver,
constraints generated unwisely would still hurt its performance,
sometimes even making it run forever.

\paragraph{Bounded constraint size.}
It is not a good idea to naively analyze and generate constraints
interprocedurally, for example,  across the whole Linux kernel.
The size of the path constraint would grow exponentially, which is
unnecessary and hard to solve.  To achieve scalability, \sys should
choose an appropriate program granularity.

\sys also needs to handle complex program constructs such as loops
and pointer arithmetic appropriately.  The generated constraints
should be able to catch common integer errors while being solvable
in a reasonable amount of time.

\paragraph{Idioms for faster solving.}
We notice that some operations like division would significantly
slow down the constraint solver~\cite{brummayer:perf}, most of which
are used in integer error checks like $\cc{UINT_MAX} /_u x > y$.
\sys should recognize these idioms and generate constraints that
are easier to solve.

\fi
