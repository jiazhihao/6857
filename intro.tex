\section{Introduction}
\label{s:intro}

Integers are fundamental data types in systems code.  Unlike
mathematical integers, they use a fixed number of bits to represent
integral values bounded by a predefined range.  An operation producing
a value that does not fall into the range may cause lost or
misinterpreted data.  It may introduce logic errors and security
vulnerabilities if the consequence is unexpected.

\autoref{f:vmwgfx} shows an example of integer errors in a Linux
kernel driver.  The size for allocating \cc{rects} is computed from
the multiplication of \cc{arg->num_outputs}, a variable controlled
by userspace, and \cc{sizeof(struct drm_vmw_rect)}, the constant
$16$.  Since both \cc{arg->num_outputs} and \cc{rects} are 32-bit
integers, their values are within the range from 0 to $2^{32} - 1$.

If a malicious user supplies a big integer, say \cc{0xf0000000},
for the variable \cc{arg->num_outputs}, the multiplication would
overflow the 32-bit integer range and wrap to \cc{0xf0000000 * 16
= 0} for the allocation size.  Then \cc{rects} is a zero-length
array, and the access \cc{rects[i]} is out of bounds, which may
cause a kernel panic.

[[[they are serious.]
[[[more security impacts --- buffer overflow, information leak,
privilege escalation]]]
Integer errors have emerged as a threat to system security.  A 2007
study of the CVE list~\cite{christey:vuln} suggests that they ``are
number 2 for OS vendor advisories''.  A recent survey~\cite{chen:kbugs}
confirms the finding, where integer errors account for more than
one third of the reported Linux kernel vulnerabilities from 2010
to early 2011 that can be misused to corrupt the kernel and break
its integrity.

[[[won't triggered under inputs, but impose a security threat
on abnormal input or rarely executed code --- testing is hard.
]]]

[[[non-trivial to make the code right. patches may be incorrect.]]]
--- precise bit-level reasoning of bounds required.

\begin{figure}
\begin{Verbatim}[commandchars=\\\{\}]
\PY{c+c1}{// "data" is passed from userspace.}
\PY{k}{struct} \PY{n}{drm\PYZus{}vmw\PYZus{}update\PYZus{}layout\PYZus{}arg} \PY{o}{*}\PY{n}{arg} \PY{o}{=} \PY{n}{data}\PY{p}{;}
\PY{k}{struct} \PY{n}{drm\PYZus{}vmw\PYZus{}rect} \PY{o}{*}\PY{n}{rects} \PY{o}{=} \PY{l+m+mi}{0}\PY{p}{;}
\PY{k+kt}{unsigned} \PY{n}{size}\PY{p}{;}
\PY{k+kt}{int} \PY{n}{i}\PY{p}{,} \PY{n}{ret} \PY{o}{=} \PY{l+m+mi}{0}\PY{p}{;}
\PY{c+c1}{// Check arg->num\PYZus{}outputs.}
\PY{k}{if} \PY{p}{(}\PY{o}{!}\PY{n}{arg}\PY{o}{-}\PY{o}{>}\PY{n}{num\PYZus{}outputs}\PY{p}{)}
	\PY{k}{goto} \PY{n}{out}\PY{p}{;}
\PY{c+c1}{// Compute allocation size for "rects".}
\PY{c+c1}{// sizeof(struct drm\PYZus{}vmw\PYZus{}rect) = 16.}
\PY{n}{size} \PY{o}{=} \PY{n}{arg}\PY{o}{-}\PY{o}{>}\PY{n}{num\PYZus{}outputs} \PY{o}{*} \PY{k}{sizeof}\PY{p}{(}\PY{k}{struct} \PY{n}{drm\PYZus{}vmw\PYZus{}rect}\PY{p}{)}\PY{p}{;}
\PY{c+c1}{// Allocate memory.}
\PY{n}{rects} \PY{o}{=} \PY{n}{kzalloc}\PY{p}{(}\PY{n}{size}\PY{p}{,} \PY{n}{GFP\PYZus{}KERNEL}\PY{p}{)}\PY{p}{;}
\PY{k}{if} \PY{p}{(}\PY{n}{unlikely}\PY{p}{(}\PY{o}{!}\PY{n}{rects}\PY{p}{)}\PY{p}{)} \PY{p}{\PYZob{}}
	\PY{n}{ret} \PY{o}{=} \PY{o}{-}\PY{n}{ENOMEM}\PY{p}{;}
	\PY{k}{goto} \PY{n}{out}\PY{p}{;}
\PY{p}{\PYZcb{}}
\PY{c+c1}{// Validate input data.}
\PY{k}{for} \PY{p}{(}\PY{n}{i} \PY{o}{=} \PY{l+m+mi}{0}\PY{p}{;} \PY{n}{i} \PY{o}{<} \PY{n}{arg}\PY{o}{-}\PY{o}{>}\PY{n}{num\PYZus{}outputs}\PY{p}{;} \PY{o}{+}\PY{o}{+}\PY{n}{i}\PY{p}{)} \PY{p}{\PYZob{}}
	\PY{k}{if} \PY{p}{(}\PY{n}{rects}\PY{p}{[}\PY{n}{i}\PY{p}{]}\PY{o}{-}\PY{o}{>}\PY{n}{x} \PY{o}{<} \PY{l+m+mi}{0} \PY{o}{|}\PY{o}{|} \PY{n}{rects}\PY{p}{[}\PY{n}{i}\PY{p}{]}\PY{o}{-}\PY{o}{>}\PY{n}{y} \PY{o}{<} \PY{l+m+mi}{0}\PY{p}{)} \PY{p}{\PYZob{}}
		\PY{n}{ret} \PY{o}{-}\PY{n}{EINVAL}\PY{p}{;}
		\PY{k}{goto} \PY{n}{out}\PY{p}{;}
	\PY{p}{\PYZcb{}}
\PY{p}{\PYZcb{}}
\PY{c+c1}{// Clean up. }
\PY{n+nl}{out:}
	\PY{n}{kfree}\PY{p}{(}\PY{n}{rects}\PY{p}{)}\PY{p}{;}
	\PY{k}{return} \PY{n}{ret}\PY{p}{;}
\end{Verbatim}

\caption{An integer error (simplified) in the VMware graphics driver
for the Linux kernel that would result in out-of-bounds reads.}
\label{f:vmwgfx}
\end{figure}
