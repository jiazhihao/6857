\section{NaN Integer Semantics}
\label{s:miti}

\autoref{s:eval} shows that a major source of integer errors is due
to unanticipated integer wraparound.  To mitigate the issue, we
propose a new integer family with NaN semantics: the value enters
and stays in NaN state once an integer error occurs.
%
This section will illustrate how NaN integers help automate integer
error checking.

The basic usage is to add a type qualifier \cc{nan}
to an actual integer type, such as \cc{nan int}. The type conversion
rules for NaN integers are as follows.
\begin{CompactItemize}
\item
An integer of type $T$ will be automatically promoted to $\cc{nan}\ T$
when used with an integer of type $\cc{nan}\ T$.
\item
The resulting type of an arithmetic or comparison operation with
operands of type $\cc{nan}\ T$ is also $\cc{nan}\ T$.
\end{CompactItemize}

We implement NaN integers by modifying the Clang compiler.
The compiler inserts overflow checks for every arithmetic or
comparison operation of type $\cc{nan}\ T$, and sets the result to
NaN state if any source operand is NaN state, or an integer error occurred
during the operation; otherwise the operation follows standard C rules.
We choose the maximum value as NaN state for unsigned integers,
and the minimum for signed integers.

The compiler generates efficient overflow detection instructions
for these checks. On x86, for example,
the compiler inserts one \cc{jno} instruction after the 
multiplication, which jumps in case of no overflow.

%\noindent
The NaN integer API consists of two functions:
\begin{CompactItemize}
\item
$\cc{bool}\ \cc{isnan}(\cc{nan}\ T\; x)$ \hfill (NaN testing) \\
returns true if $x$ is NaN state, or false otherwise.
\item
$T\ \cc{unnan}(\cc{nan}\ T\; x,\; T\; v)$ \hfill (Type casting) \\
returns $v$ if $x$ is NaN state, or $(T)x$ otherwise.
\end{CompactItemize}

To demonstrate the use of the API, we modified libc's \cc{malloc}
to take an \cc{nan size_t} as the allocation size.
\begin{Verbatim}[commandchars=\\\{\},codes={\catcode`\$=3\catcode`\^=7\catcode`\_=8}]
\PY{k+kt}{void} \PY{o}{*}\PY{n+nf}{malloc}\PY{p}{(}\PY{k}{nan} \PY{k+kt}{size\PYZus{}t} \PY{n}{size}\PY{p}{)}
\PY{p}{\PYZob{}}
    \PY{k+kt}{size\PYZus{}t} \PY{n}{bytes} \PY{o}{=} \PY{k}{unnan}\PY{p}{(}\PY{n}{size}\PY{p}{,} \PY{n}{SIZE\PYZus{}MAX}\PY{p}{)}\PY{p}{;}
    \PY{k}{return} \PY{n}{libc\PYZus{}malloc}\PY{p}{(}\PY{n}{bytes}\PY{p}{)}\PY{p}{;}
\PY{p}{\PYZcb{}}
\end{Verbatim}

The modified \cc{malloc} returns \cc{NULL} if the allocation size
overflowed.  To do so, it sets the allocation size to a large value
\cc{SIZE_MAX} if \cc{size} overflowed, which will cause
\cc{libc_malloc(SIZE_MAX)} to return \cc{NULL}.

With the NaN-ized \cc{malloc}, developers do not have to write any overflow checks
when calculating the allocation size.
An example usage is as follows.
\begin{Verbatim}[commandchars=\\\{\},codes={\catcode`\$=3\catcode`\^=7\catcode`\_=8}]
\PY{n}{nan} \PY{k+kt}{size\PYZus{}t} \PY{n}{n} \PY{o}{=} \PY{p}{.}\PY{p}{.}\PY{p}{.}\PY{p}{;}
\PY{n}{nan} \PY{k+kt}{size\PYZus{}t} \PY{n}{size} \PY{o}{=} \PY{p}{.}\PY{p}{.}\PY{p}{.}\PY{p}{;}
\PY{k+kt}{void} \PY{o}{*}\PY{n}{p} \PY{o}{=} \PY{n}{malloc}\PY{p}{(}\PY{n}{n} \PY{o}{*} \PY{n}{size}\PY{p}{)}\PY{p}{;}
\PY{k}{if} \PY{p}{(}\PY{o}{!}\PY{n}{p}\PY{p}{)} \PY{p}{\PYZob{}} \PY{p}{.}\PY{p}{.}\PY{p}{.} \PY{p}{\PYZcb{}}
\end{Verbatim}


\begin{figure}
\centering
\begin{tabular}{lr@{$\pm$}lr@{$\pm$}l}\toprule
& \multicolumn{2}{c}{w/o \cc{malloc}} & \multicolumn{2}{c}{w \cc{malloc}} \\
\midrule
no check
&  3.00 & 0.01 & 79.03  & 0.01
\\
manual check
& 24.01 & 0.01 & 104.04 & 0.03
\\
NaN integer check
& 4.05 & 0.17 & 82.03 & 0.05
\\
\bottomrule
\end{tabular}
\caption{Performance overhead of \cc{n * size},
using manual overflow check \cc{x != 0 \&\& y > UINT_MAX / x} and
automated check via NaN integers, for both w/o and w/ \cc{malloc}
cases, measured in cycles per operation over $10^6$ back-to-back
operations, averaged over 1,000 runs.}
\label{f:data:nan-micro}
\end{figure}

To show how NaN integers help simplify overflow checks,
consider the following (simplified) code snippet from the
\cc{mq_attr_ok} function in Linux kernel:
%
\begin{Verbatim}[commandchars=\\\{\},codes={\catcode`\$=3\catcode`\^=7\catcode`\_=8}]
\PY{c+cm}{/* overflow checks for}
\PY{c+cm}{ *   maxmsg * msgsize + maxmsg * sizeof(void *)}
\PY{c+cm}{ * assume}
\PY{c+cm}{ *   0 \PYZlt{} maxmsg $\leq$ $2^{15}$}
\PY{c+cm}{ *   0 \PYZlt{} msgsize $\leq$ $2^{31}$}
\PY{c+cm}{ */}
\PY{k}{if} \PY{p}{(}\PY{n}{msgsize} \PY{o}{\PYZgt{}} \PY{n}{ULONG\PYZus{}MAX} \PY{o}{/} \PY{n}{maxmsg}\PY{p}{)}
    \PY{k}{return} \PY{l+m+mi}{0}\PY{p}{;}
\PY{k}{if} \PY{p}{(}\PY{n}{maxmsg} \PY{o}{*} \PY{p}{(}\PY{n}{msgsize} \PY{o}{+} \PY{k}{sizeof}\PY{p}{(}\PY{k+kt}{void} \PY{o}{*}\PY{p}{)}\PY{p}{)}
    \PY{o}{\PYZlt{}} \PY{n}{maxmsg} \PY{o}{*} \PY{n}{msgsize}\PY{p}{)}
    \PY{k}{return} \PY{l+m+mi}{0}\PY{p}{;}
\PY{k}{return} \PY{l+m+mi}{1}\PY{p}{;}
\end{Verbatim}

%
The developer wrote two separate overflow checks for
\cc{maxmsg * msgsize + maxmsg * sizeof(void *)}:
one check for the first multiplication \cc{maxmsg * msgsize}, and
the other for the addition;
note that the second multiplication \cc{maxmsg * sizeof(void *)}
never overflows 32 bits since \cc{maxmsg} is at most $2^{15}$.

With NaN integers, a developer can directly use \cc{isnan}
without worrying about intermediate overflows:
\begin{Verbatim}[commandchars=\\\{\},codes={\catcode`\$=3\catcode`\^=7\catcode`\_=8}]
\PY{k}{static} \PY{k+kt}{int} \PY{n+nf}{mq\PYZus{}attr\PYZus{}ok}\PY{p}{(}\PY{p}{.}\PY{p}{.}\PY{p}{.}\PY{p}{,} \PY{k}{struct} \PY{n}{mq\PYZus{}attr} \PY{o}{*}\PY{n}{attr}\PY{p}{)}
\PY{p}{\PYZob{}}
    \PY{c+cm}{/*attr-\PYZgt{}mq\PYZus{}msgsize \PYZgt{} 0 and attr-\PYZgt{}mq\PYZus{}maxmsg \PYZgt{} 0*/}
    \PY{k+kt}{nan} \PY{k+kt}{unsigned} \PY{k+kt}{long} \PY{n}{msg} \PY{o}{=} \PY{n}{attr}\PY{o}{-}\PY{o}{\PYZgt{}}\PY{n}{mq\PYZus{}maxmsg}\PY{p}{;}
    \PY{k+kt}{nan} \PY{k+kt}{unsigned} \PY{k+kt}{long} \PY{n}{size} \PY{o}{=} \PY{n}{attr}\PY{o}{-}\PY{o}{\PYZgt{}}\PY{n}{mq\PYZus{}msgsize}\PY{p}{;}
    \PY{k}{if} \PY{p}{(}\PY{n+nf}{isnan}\PY{p}{(}\PY{n}{msg} \PY{o}{*} \PY{p}{(}\PY{n}{size} \PY{o}{+} \PY{k}{sizeof}\PY{p}{(}\PY{k}{struct} \PY{n}{msg\PYZus{}msg}\PY{o}{*}\PY{p}{)}\PY{p}{)}\PY{p}{)}\PY{p}{)}
        \PY{k}{return} \PY{l+m+mi}{0}\PY{p}{;}
    \PY{k}{return} \PY{l+m+mi}{1}\PY{p}{;}
\PY{p}{\PYZcb{}}
\end{Verbatim}


To show that the runtime overhead of NaN integers is low, we compare
the cost of a single multiplication (\cc{x * y}), as well as a multiplication
followed by a \cc{malloc} call, in three scenarios: with no overflow
check, with a manual overflow check using
\cc{x != 0 \&\& y > UINT_MAX / x}, and with an overflow check using
NaN integers.
%
\autoref{f:data:nan-micro} shows the results.
%
With a single multiplication, overflow checking using NaN integers
adds 1--3 cycles on average, and manual overflow checking adds
21--25 cycles.
%
Given the negligible overhead, we believe that it is practical
to replace manual overflow checks with NaN integers.

For compilers that do not support NaN integers, we encourage
developers to use safe allocators that perform overflow checking
internally.  Particularly, we contributed \cc{kmalloc_array(n,
size)} for array allocation, which checks overflow for $\cc{n}
\times_u \cc{size}$. This function has been incorporated in the Linux kernel
since v3.4-rc1.

\if 0
Below is an example check in the infiniband driver.
\input{code/ib-check}
In the check, \cc{wqe_size}, \cc{wr_count}, and \cc{sqe_count} are
all directly from user space, and thus their values can be any
(32-bit) integers.  Carefully crafted values would overflow the
multiplications and addition at the right-hand side and bypass
check.  To avoid overflows, one would need to write additional
checks, such as:
\input{code/ib-more}
With NaN integers, one needs to change the type of
\cc{wqe_size}, \cc{wr_count}, and \cc{sqe_count} to NaN integers,
and wrap the original check with $\cc{nan}(\dots, \cc{true})$.
\input{code/ib-nan}
\fi

\if 0

\subsection{Compiler Checking}

It would be useful to have the compiler check and warn against
integer errors, so that developers can fix these bugs
at compile time.  Ideally, the check should not slow down
the compilation too much, and the false error rate should be kept low,
so that developers are willing to enable the check.
%
Due to the above two concerns, one cannot directly integrate
\sys's solver-based approach into a compiler.  A pragmatic workaround
could be to match and warn against the common error patterns
summarized in \autoref{s:common}, which can be done efficiently
with a low false error rate.

Decent C compilers have some support for detecting integer error
patterns.
%
GCC provides \cc{-Wtype-limits} to warn if a comparison if always
true or false due to type range limits.
%however this
%option is not turned on even with \cc{-Wall}, probably because of
%its high false rate.
%
Clang has a similar option called \cc{-Wtautological-compare}.
%
We test these features by running latest versions of both compilers
against three simple error patterns extracted from the bugs discovered
by \sys.  The result as shown in \autoref{f:cmp-stat}.  GCC cannot
find any of them with \cc{-Wall}; it can find two with \cc{-Wextra}
or \cc{-Wtype-limits}, but these options are usually turned off by
developers due to high false error rate.  Clang can find one of them
without any special options.
%
%Neither of the compilers can detect the third error pattern, which
%is simply a variation of the second one.
%
We plan to push our work on integer error patterns into these C
compilers and improve their compile-time checking mechanism.

\begin{figure}
\centering
\begin{tabular}{lll} \toprule
  & GCC~4.6.1 & Clang~3.0 \\ \midrule

$\cc{uint} < 0$
& \ok \cc{-Wextra} & \ok \\
$\cc{uchar} < 0$
& \ok \cc{-Wextra} & \\
$\cc{int}\ r = \cc{uchar}; r < 0$
\\

\bottomrule
\end{tabular}

\caption{Applying GCC and Clang to checking three simple integer
error patterns.  GCC can detect two, but only when \cc{-Wextra} is
given.  Clang can detect one without any special options.}
\label{f:cmp-stat}
\end{figure}

\subsection{Replacing Allocation Calls}

Integer errors that affect allocation size are particularly dangerous,
as discussed in \autoref{s:rank}.
We plan to develop a tool to prevent integer errors
in such cases at run time.
The idea is to extend the C compiler
to insert checks for integer operations that calculate allocation sizes.
If any of the checks fails at run time,
the inserted code forces the corresponding allocation call to return
\cc{NULL}, which we assume is handled by existing code
as out of memory.

\fi

\if 0
The grsecurity patch~\cite{grsecurity}
for the Linux kernel addresses the issue by replacing
\cc{kmalloc} with a macro of the same name.
%
\begin{Verbatim}[commandchars=\\\{\}]
\PY{c+cp}{\PYZsh{}}\PY{c+cp}{define kmalloc(x, y)}                           \PYZbs{}
\PY{p}{(}\PY{p}{\PYZob{}}                                              \PYZbs{}
    \PY{k+kt}{void} \PY{o}{*}\PY{n}{\PYZus{}\PYZus{}\PYZus{}retval}\PY{p}{;}                            \PYZbs{}
    \PY{n}{intoverflow\PYZus{}t} \PY{n}{\PYZus{}\PYZus{}\PYZus{}x} \PY{o}{=} \PY{p}{(}\PY{n}{intoverflow\PYZus{}t}\PY{p}{)}\PY{n}{x}\PY{p}{;}      \PYZbs{}
    \PY{k}{if} \PY{p}{(}\PY{n}{WARN}\PY{p}{(}\PY{n}{\PYZus{}\PYZus{}\PYZus{}x} \PY{o}{>} \PY{n}{ULONG\PYZus{}MAX}\PY{p}{,} \PY{l+s}{"}\PY{l+s}{overflow}\PY{l+s+se}{\PYZbs{}n}\PY{l+s}{"}\PY{p}{)}\PY{p}{)}   \PYZbs{}
        \PY{n}{\PYZus{}\PYZus{}\PYZus{}retval} \PY{o}{=} \PY{n+nb}{NULL}\PY{p}{;}                       \PYZbs{}
    \PY{k}{else}                                        \PYZbs{}
        \PY{n}{\PYZus{}\PYZus{}\PYZus{}retval} \PY{o}{=} \PY{n}{kmalloc}\PY{p}{(}\PY{p}{(}\PY{k+kt}{size\PYZus{}t}\PY{p}{)}\PY{n}{\PYZus{}\PYZus{}\PYZus{}x}\PY{p}{,} \PY{p}{(}\PY{n}{y}\PY{p}{)}\PY{p}{)}\PY{p}{;} \PYZbs{}
    \PY{n}{\PYZus{}\PYZus{}\PYZus{}retval}\PY{p}{;}                                  \PYZbs{}
\PY{p}{\PYZcb{}}\PY{p}{)}
\end{Verbatim}

%
Here \cc{intoverflow_t} is defined as the 128-bit unsigned integer
type.  The trick is that, assuming the developer often writes code
like \cc{kmalloc(count * size)}, with the macro of the same name
the C preprocessor will silently change the call to
\cc{kmalloc((intoverflow_t)count * size)}, where \cc{count} is first
promoted to 128 bits for the computation.  This should be large
enough to detect most integer errors.

However, the trick does not work if the call is written in forms
like \cc{kmalloc(header_size + count * size)}, or the allocation
size is already computed before being passed to \cc{kmalloc}, such
as in \autoref{f:bridge}.
\fi
