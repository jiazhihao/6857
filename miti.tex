\section{Mitigation}
\label{s:miti}

This section discusses mitigate strategies for integer errors in
software.

\subsection{Compiler Checking}

It would be useful to have the compiler check and warn against
integer errors, so that developers can easily notice these bugs and
fix them at compile time.  Ideally, the check should not slow down
the compilation too much, and the false rate should be kept low,
so that developers are willing to enable the check.

Due to the above two concerns, it is unlikely to directly integrate
our solver-based approach into a compiler.  A pragmatic workaround
could be to match and warn against the common error patterns
summarized in \autoref{s:common}, which can be done efficiently
with a low false rate.

Decent C compilers have some support for detecting integer error
patterns.
%
GCC provides \cc{-Wtype-limits} to warn if a comparison if always
true or false due to type range limits.
%however this
%option is not turned on even with \cc{-Wall}, probably because of
%its high false rate.
%
Clang has a similar option called \cc{-Wtautological-compare}.

We test these features by running latest versions of both compilers
against three simple error patterns extracted from the bugs discovered
by \sys.  The result as shown in \autoref{f:cmp-stat}.  GCC cannot
find any of them with \cc{-Wall}; it can find two with \cc{-Wextra}
or \cc{-Wtype-limits}, but these options are usually turned off by
developers due to high false rate.  Clang can find one of them
without any special option.
%
%Neither of the compilers can detect the third error pattern, which
%is simply a variation of the second one.
%
We plan to push our work on integer error patterns into these C
compilers and improve their compile-time checking mechanism.

\begin{figure}
\centering
\begin{tabular}{lll} \toprule
  & GCC~4.6.1 & Clang~3.0 \\ \midrule

$\cc{uint} < 0$
& \ok \cc{-Wextra} & \ok \\
$\cc{uchar} < 0$
& \ok \cc{-Wextra} & \\
$\cc{int}\ r = \cc{uchar}; r < 0$
\\

\bottomrule
\end{tabular}

\caption{Applying GCC and Clang to checking three simple integer
error patterns.  GCC can detect two, but only when \cc{-Wextra} is
given.  Clang can detect one without any special option.}
\label{f:cmp-stat}
\end{figure}

\subsection{Alternative Integer Semantics}

Wrapping is the default behavior in languages like Java.
GCC provides the option \cc{-fwrapv} to enforce wrapping semantics
for signed integers in C.  The Linux kernel is compiled with a
slightly different option \cc{-fno-strict-overflow}, to prevent GCC
from optimizing away checks with signed integers (e.g., $x + 1 <
x$) and pointers (e.g., $p + \mathit{len} < p$).
These options reduce the risk of unwanted optimizations.

A different strategy is trapping.  Ada raises an exception at run
time for out-of-range values.  For C, GCC provides the option
\cc{-ftrapv} to insert overflow checks for signed integer operations
in the generated code; any violation will cause a trap at run time.
This is extended by Clang via \cc{-fcatch-undefined-behavior} for
more undefined integer operations such as oversized shifts, and by
other compiler patches~\cite{brumley:rich, ioc} even for unsigned
integer overflows.  They are useful for detecting integer errors
at run time.

The main concern of enabling trapping semantics for production code
is that it introduces false alarms at run time, while there is no
convenient way for developers to provide fine-grained error handlers
if such an alarm is raised.  It also incurrs noticeable runtime
overhead~\cite{ioc}, given the prevalence of integer operations.

\subsection{Using an Integer Library}

Another approach to secure the code is to enforce the use of an integer
library with error checks, such as CERT's
IntegerLib~\cite[INT03-C]{seacord:secure-c}.  For example, developers
change their code by calling a library function $\cc{addsl}(x, y)$
for addition of two signed long integers $x$ and $y$.  The library
then performs sanity checks at run time, and invokes a preset error
handler if an integer error happens.

SafeInt~\cite{safeint}, a C++ template class widely used in Microsoft
products and Mozilla Firefox, eases the adoption and allows to write
the intuitive expression $x + y$ via operator overloading in C++;
developers need to declare $x$ and $y$ as the \cc{SafeInt<long>}
type.

The underlying integer library is trusted to be implemented
correctly.  Unfortunately, integer errors have recently been
discovered in both IntegerLib and SafeInt~\cite{ioc}.
Below is IntegerLib's \cc{addsl} implementation.
%
\begin{Verbatim}[commandchars=\\\{\},codes={\catcode`\$=3\catcode`\^=7\catcode`\_=8}]
\PY{k+kt}{signed} \PY{k+kt}{long} \PY{n+nf}{addsl}\PY{p}{(}\PY{k+kt}{signed} \PY{k+kt}{long} \PY{n}{lhs}\PY{p}{,} \PY{k+kt}{signed} \PY{k+kt}{long} \PY{n}{rhs}\PY{p}{)} \PY{p}{\PYZob{}}
  \PY{n}{errno} \PY{o}{=} \PY{l+m+mi}{0}\PY{p}{;}
  \PY{k}{if} \PY{p}{(} \PY{p}{(}\PY{p}{(}\PY{p}{(}\PY{n}{lhs}\PY{o}{+}\PY{n}{rhs}\PY{p}{)}\PY{o}{\PYZca{}}\PY{n}{lhs}\PY{p}{)} \PY{o}{&} \PY{p}{(}\PY{p}{(}\PY{n}{lhs}\PY{o}{+}\PY{n}{rhs}\PY{p}{)}\PY{o}{\PYZca{}}\PY{n}{rhs}\PY{p}{)}\PY{p}{)}
        \PY{o}{>}\PY{o}{>} \PY{p}{(}\PY{k}{sizeof}\PY{p}{(}\PY{k+kt}{int}\PY{p}{)}\PY{o}{*}\PY{n}{CHAR\PYZus{}BIT}\PY{o}{-}\PY{l+m+mi}{1}\PY{p}{)} \PY{p}{)} \PY{p}{\PYZob{}}
    \PY{n}{error\PYZus{}handler}\PY{p}{(}\PY{l+s}{"}\PY{l+s}{OVERFLOW ERROR}\PY{l+s}{"}\PY{p}{,} \PY{n+nb}{NULL}\PY{p}{,} \PY{n}{EOVERFLOW}\PY{p}{)}\PY{p}{;}
    \PY{n}{errno} \PY{o}{=} \PY{n}{EINVAL}\PY{p}{;}
  \PY{p}{\PYZcb{}}
  \PY{k}{return} \PY{n}{lhs}\PY{o}{+}\PY{n}{rhs}\PY{p}{;}
\PY{p}{\PYZcb{}}
\end{Verbatim}

We believe that there are two issues in the above code.
%
First, \cc{sizeof(int)} should be \cc{sizeof(signed long)}, otherwise
the code would trigger false alarms on a 64-bit system.  Second,
the code relies on undefined behavior, i.e., signed overflow of
\cc{lhs+rhs}, to defend against overflow.  A compiler may rewrite
the code in unexpected ways.


\subsection{Replacing Allocation Calls}

Integer errors that affect allocation size is particularly dangerous,
as discussed in \autoref{s:rank}.  The grsecurity patch~\cite{grsecurity}
for of the Linux kernel addresses the issue by replacing
\cc{kmalloc} with a macro of the same name.
%
\begin{Verbatim}[commandchars=\\\{\}]
\PY{c+cp}{\PYZsh{}}\PY{c+cp}{define kmalloc(x, y)}                           \PYZbs{}
\PY{p}{(}\PY{p}{\PYZob{}}                                              \PYZbs{}
    \PY{k+kt}{void} \PY{o}{*}\PY{n}{\PYZus{}\PYZus{}\PYZus{}retval}\PY{p}{;}                            \PYZbs{}
    \PY{n}{intoverflow\PYZus{}t} \PY{n}{\PYZus{}\PYZus{}\PYZus{}x} \PY{o}{=} \PY{p}{(}\PY{n}{intoverflow\PYZus{}t}\PY{p}{)}\PY{n}{x}\PY{p}{;}      \PYZbs{}
    \PY{k}{if} \PY{p}{(}\PY{n}{WARN}\PY{p}{(}\PY{n}{\PYZus{}\PYZus{}\PYZus{}x} \PY{o}{>} \PY{n}{ULONG\PYZus{}MAX}\PY{p}{,} \PY{l+s}{"}\PY{l+s}{overflow}\PY{l+s+se}{\PYZbs{}n}\PY{l+s}{"}\PY{p}{)}\PY{p}{)}   \PYZbs{}
        \PY{n}{\PYZus{}\PYZus{}\PYZus{}retval} \PY{o}{=} \PY{n+nb}{NULL}\PY{p}{;}                       \PYZbs{}
    \PY{k}{else}                                        \PYZbs{}
        \PY{n}{\PYZus{}\PYZus{}\PYZus{}retval} \PY{o}{=} \PY{n}{kmalloc}\PY{p}{(}\PY{p}{(}\PY{k+kt}{size\PYZus{}t}\PY{p}{)}\PY{n}{\PYZus{}\PYZus{}\PYZus{}x}\PY{p}{,} \PY{p}{(}\PY{n}{y}\PY{p}{)}\PY{p}{)}\PY{p}{;} \PYZbs{}
    \PY{n}{\PYZus{}\PYZus{}\PYZus{}retval}\PY{p}{;}                                  \PYZbs{}
\PY{p}{\PYZcb{}}\PY{p}{)}
\end{Verbatim}

%
Here \cc{intoverflow_t} is defined as the 128-bit unsigned integer
type.  The trick is that, assuming the developer often writes code
like \cc{kmalloc(count * size)}, with the macro of the same name
the C preprocessor will silently change the call to
\cc{kmalloc((intoverflow_t)count * size)}, where \cc{count} is first
promoted to 128 bits for the computation.  This should be large
enough to detect most integer errors.

However, the trick does not work if the call is written in forms
like \cc{kmalloc(header_size + count * size)}, or the allocation
size is already computed before being passed to \cc{kmalloc}, such
as in \autoref{f:bridge}.


\subsection{Language Design}

Developers would benefit from language support for specifying precise
lower and upper bounds of variables.  For example, Ada allows to
define a ranged subtype, e.g., integers from 0 to 9.  The runtime
will raise an exception for any attempt to storing an out-of-range
value to variables of that subtype, and developers are responsible
for handling the exception.  There is a similar proposal that adds
ranged integers to C~\cite{ranged-c}.

Another option is to use infinitely ranged integers, aka bignum,
if the code is not performance critical.  Scripting languages such
as Python and Ruby have built-in bignum support.  Using bignum is
generally more immune to integer errors since it does not wrap,
although the underlying bignum implementation, often written in C,
may contain vulnerabilities.  CVE-2011-0188 is a recent example in
Ruby's bignum, where an improper truncation may cause a buffer
overflow.
%shown as below.
%\begin{Verbatim}[commandchars=\\\{\},codes={\catcode`\$=3\catcode`\^=7\catcode`\_=8}]
\PY{g+gd}{--- ext/bigdecimal/bigdecimal.c 2010/09/29 08:55:17 ...}
\PY{g+gi}{+++ ext/bigdecimal/bigdecimal.c 2011/03/01 04:40:49 ...}
\PY{g+gu}{@@ -2237,8 +2237,8 @@}
 VP\PYZus{}EXPORT void *
 VpMemAlloc(size\PYZus{}t mb)
 \PYZob{}
\PY{g+gd}{-    void *p = xmalloc((unsigned int)mb);}
\PY{g+gi}{+    void *p = xmalloc(mb);}
     if (!p) \PYZob{}
         VpException(VP\PYZus{}EXCEPTION\PYZus{}MEMORY, ...);
     \PYZcb{}
     memset(p, 0, mb);
\end{Verbatim}


We suspect that removing unsigned types from the language would
not help reduce the risk of integer errors.  For example, Java
supports only signed integers.  However, its binary search
implementation contained an integer bug, which was not fixed until
JDK~6~\cite{java-bsearch}.  The buggy code calculates the midpoint
using $(\cc{low} + \cc{high})/2$, which will become negative for
large $\cc{low}$ and $\cc{high}$.  A correct version is $\cc{low}
+ (\cc{high}-\cc{low})/2$.  In the C language one may simply declare
$\cc{low}$ and $\cc{high}$ as unsigned integers to avoid the problem.

\if 0
range annotations?  something like?
\begin{Verbatim}
#define RANGE(lo, hi) __attribute__((range(lo, hi)))
#define MAX_ELTS 1024

struct X {
	unsigned int nelts RANGE(1, MAX_ELTS);
	struct Y *elts;
};
\end{Verbatim}
Then you can do instrumentation on stores of nelts.
Better runtime checks \& good for static analysis.
\fi
